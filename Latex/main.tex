% !TeX root = main.tex
\documentclass[12pt, a4paper, twoside]{report}   % Tamaño de papel, de fuente y márgenes

\usepackage{shellesc}
 
\usepackage[spanish,es-ucroman]{babel}  % Idioma y codificación, para poder usar
\usepackage[T1]{fontenc}
% \usepackage[utf8x]{inputenc} % tildes, eñes, etc., sin problemas\usepackage{newunicodechar}
\usepackage{array}

% Para tablas dinámicas
\usepackage{tabularx}

% Para fijar el interlineado
\usepackage{setspace} 

\usepackage{fancyhdr}                   % Para los encabezados y pies
\usepackage{lipsum}                     % Para generar texto como contenido de prueba, se puede eliminar en la memoria final
\usepackage{helvet}                     % Para fuentes (Using common PostScript fonts with LaTeX)
\usepackage{wallpaper}                  % Para introducir el PDF con la portada EPSJ
\usepackage[absolute,overlay]{textpos}  % 
\usepackage{etoolbox}                   % Herramientas de logica, variables y condicionales y bibliografía
\usepackage[numbers]{natbib}            % Para gestionar la bibliografia
\usepackage{csquotes}                   % 
\usepackage{hyperref}                   % Para URL como enlaces
\usepackage{booktabs}                   % Para tablas con mejor apariencia
\usepackage{listings}                   % Para incluir listados de código

% Para los emojis, debes compilar el paquete
\usepackage{twemojis}

\usepackage{xcolor}                     % El paquete parte de las facilidades básicas del paquete de color 
\usepackage[ruled,vlined]{algorithm2e}  % https://ctan.javinator9889.com/macros/latex/contrib/algorithm2e/doc/algorithm2e.pdf
\usepackage[acronym]{glossaries}        %
\usepackage{pdflscape}                  %

% 
\usepackage{ifluatex}
\ifluatex
  \usepackage{pdftexcmds}
  \makeatletter
  \let\pdfstrcmp\pdf@strcmp
  \let\pdffilemoddate\pdf@filemoddate
  \makeatother
\fi

% Para las imágenes
\usepackage{svg}                  % Enable --shell-escape for external commands
% \setsvg{inkscapeexe=inkscape, inkscapeopt=-z -D}     
\usepackage{wrapfig}                   
\usepackage{float}                     
\raggedbottom 

  
% Gestion de los tamaños de las páginas
\usepackage{geometry}                   % https://www.ctan.org/pkg/geometry
\usepackage{fancyhdr}                   % https://www.ctan.org/pkg/fancyhdr

\usepackage{setspace}                   %for doublespacing
\doublespacing %double spacing
\geometry{
  a4paper,
  left=1.5in,
  right=1.5in,
  top=1in,
  bottom=1in,
  margin=2.5cm,
  heightrounded,
}
 
% \usepackage[acronym,nomain,toc,automake]{glossaries-extra}
% \usepackage[acronym,toc=true,nomain,xindy]{glossaries-extra}
% \setabbreviationstyle[acronym]{long-short} 

\usepackage{csvsimple}                  % https://www.ctan.org/pkg/csvsimple
\usepackage{amsmath}                    % https://www.ctan.org/pkg/amsmath
\usepackage{actuarialangle}
\usepackage{amssymb}
\usepackage{mathtools}
\usepackage{commath}
\usepackage{tabularx}
\usepackage{bookmark} 
\usepackage{caption}

% VTimeline
\usepackage{charter} 
\usepackage{environ}
\usepackage{tikz}
\usetikzlibrary{calc,matrix}

\usepackage{amsthm}
\newtheorem{theorem}{Teoremas}[section]
\newtheorem{lemma}[theorem]{Lemma}

\usepackage{enumitem}
\setlist{nolistsep}

\usepackage{tikz} 
\usetikzlibrary{calc}

\newcommand{\addterm}[7][]{%
  \newglossaryentry{#2}{name={\textit{#3} (angl.\ \textit{#5})},
     first={\textit{#3} (angl.\ \textit{#5})},
     firstplural={\textit{#4} (angl.\ \textit{#6})},
     text={#3},
     plural={#4},
     sort={#3},
     description={#7},#1}%
}


% ==== Introducir aquí el nombre del estudiante
\def\Estudiante{Antonio Mudarra Machuca}

% ==== Introducir aquí el nombre de los tutores. Si solo hay uno dejar las llaves de \TutorB vacías
\def\TutorA{Antonio Jesús Rivera Rivas}
\def\TutorB{María José del Jesus Díaz}
\def\Departamento{Departamento de informática}

% ==== Introducir aquí el título de completo y abreviado (para las cabeceras) del TFG
\def\TituloTFG{Redes neuronales adversarias en seguridad informática}
\def\TituloAbreviado{RNASI}

% ==== Introducir aquí el mes y año de presentación del TFG
\def\Fecha{septiembre de 2023}
\def\FechaPortada{septiembre, 2023}

% Configuración idioma
\renewcommand{\spanishtablename}{Tabla.}  % Título para las tablas
\renewcommand{\spanishcontentsname}{Tabla de contenidos}  % y los índices
\renewcommand{\spanishlistfigurename}{Lista de figuras}
\renewcommand{\spanishlisttablename}{Lista de tablas}

\renewcommand{\algorithmcfname}{Algoritmo}
\renewcommand{\listalgorithmcfname}{Lista de algoritmos}
\renewcommand{\lstlistingname}{Listado}
\renewcommand{\lstlistlistingname}{Lista de listados de código}

% Hoy en día la Inteligencia Artificial y sus aplicaciones están cada vez más implantadas en diferentes sistemas de nuestra sociedad. Dentro de esta disciplina el destacan el campo del Machine Learning (Aprendizaje Automático), donde como resultado de aplicar algoritmos de aprendizaje a datos se obtienen modelos que destacan por los resultados que obtienen. Gracias a su precisión estos modelos se encuentran desplegados en sistemas informáticos de gran importancia y que contralan procesos en diversos ámbitos de nuestra vida.

% Evidentemente estos sistemas pueden son vulnerables ataques de seguridad siendo los denominados modelos adversarios (basados normalmente en redes neuronales) los que suelen estar implicados tanto en estos ataques como en el posible robustecimiento de los sistemas atacados y por tanto de los modelos de aprendizaje automático en los que se basan. El objetivo de este trabajo es hacer un estudio bibliográfico del campo de los modelos adversarios aplicados a la seguridad informática. Posteriormente se elegirá un área aplicación, con sus respectivos conjuntos de datos y modelos representativos. Se realizarán experimentaciones en este área y se analizarán los resultados.

% Configuración tabularX
\newcolumntype{Y}{>{\centering\arraybackslash}X}



\hypersetup{
    colorlinks=true,
    linkcolor=blue,
    filecolor=magenta,
    urlcolor=cyan,
    pdftitle={\TituloTFG},
    pdfauthor={\Estudiante}
    bookmarks=true,
    bookmarksopen=true,
    pdfpagemode=FullScreen,
    breaklinks=true,
    citecolor=cyan,
}


\lstset{ %
    %language=delphi,                % the language of the code
    basicstyle=\linespread{0.7}\small\ttfamily,       % the size of the fonts that are used for the code
    numbers=left,                   % where to put the line-numbers
    numberstyle=\footnotesize\color{gray},  % the style that is used for the line-numbers
    stepnumber=1,                   % the step between two line-numbers. If it's 1, each line will be numbered
    numbersep=7pt,                  % how far the line-numbers are from the code
    backgroundcolor=\color{gray!5},  % choose the background color. You must add \usepackage{color}
    showspaces=false,               % show spaces adding particular underscores
    showstringspaces=true,         % underline spaces within strings
    showtabs=true,                 % show tabs within strings adding particular underscores
    frameround=fttt,
    frame=rtBL,                   % adds a frame around the code
    rulecolor=\color{black},        % if not set, the frame-color may be changed on line-breaks within not-black text (e.g. commens (green here))
    tabsize=4,                      % sets default tabsize to 2 spaces
    aboveskip=1em,
    captionpos=b,                   % sets the caption-position to bottom
    breaklines=true,                % sets automatic line breaking
    breakatwhitespace=false,        % sets if automatic breaks should only happen at whitespace
    title=\lstname,                 % show the filename of files included with \lstinputlisting;  also try caption instead of title
    keywordstyle=\bf\ttfamily,          % keyword style
    commentstyle=\color{black!60}\ttfamily,       % comment style
    stringstyle=\color{blue}\ttfamily,         % string literal style
    escapeinside={\%*}{*)}            % if you want to add a comment within your code
}

\lstset
{
    language=[LaTeX]TeX,
    breaklines=true,
    basicstyle=\tt\scriptsize,
    keywordstyle=\color{blue},
    identifierstyle=\color{gray},
    texcl=true
}
\lstset{
language=Python,
literate={á}{{\'a}}1
{ã}{{\~a}}1
{é}{{\'e}}1
{ó}{{\'o}}1
{í}{{\'i}}1
{ñ}{{\~n}}1
{¡}{{!`}}1
{¿}{{?`}}1
{ú}{{\'u}}1
{Í}{{\'I}}1
{Ó}{{\'O}}1
}

\renewcommand{\familydefault}{\sfdefault}
\setlength{\parskip}{1em}
\setlength{\headheight}{14.5pt}

% Configuración de encabezado y pie
\fancyhead[RE,LO]{{\color{gray}\Estudiante}}
\fancyhead[LE,RO]{{\color{gray}\TituloAbreviado}}
\fancyfoot[RE,LO]{{\color{gray}Escuela Politécnica Superior de Jaén}}
\fancyfoot[LE,RO]{{\color{gray}\thepage}}
\renewcommand{\footrulewidth}{1pt}


\definecolor{flashwhite}{rgb}{0.95, 0.95, 0.96}

% Emojis
% \setemojifont{EmojiOneMozilla}

% FancyHDR
\pagestyle{fancy}
\fancyhf{}

% algorithm2
\SetKwComment{Comment}{/* }{ */}

\input{macros/config-lstlisting}
%% Code by Claudio:
%% https://tex.stackexchange.com/a/197447/221452
%% Uses code by Andrew:
%% http://tex.stackexchange.com/a/28452/13304

% CAUSA CIERTOS PROBLEMAS CON {TBLR}

\makeatletter
\let\matamp=&
\catcode`\&=13
\def&{%
        \iftikz@is@matrix%
            \pgfmatrixnextcell%
        \else%
            \matamp%
        \fi%
    }
\makeatother

\newcounter{lines}
\def\endlr{\stepcounter{lines}\\}

\newcounter{vtml}
\setcounter{vtml}{0}

\newif\ifvtimelinetitle
\newif\ifvtimebottomline

\tikzset{
    description/.style={column 2/.append style={#1}},
    timeline color/.store in=\vtmlcolor,
    timeline color=red!80!black,
    timeline color st/.style={fill=\vtmlcolor,draw=\vtmlcolor},
    use timeline header/.is if=vtimelinetitle,
    use timeline header=false,
    add bottom line/.is if=vtimebottomline,
    add bottom line=false,
    timeline title/.store in=\vtimelinetitle,
    timeline title={},
    line offset/.store in=\lineoffset,
    line offset=4pt,
}

\NewEnviron{vtimeline}[1][]{%
    \setcounter{lines}{1}%
    \stepcounter{vtml}%
    \begin{tikzpicture}[column 1/.style={anchor=east},
            column 2/.style={anchor=west},
            text depth=0pt,
            text height=1ex,
            row sep=1ex,
            column sep=1em,
            #1
        ]
        \matrix(vtimeline\thevtml)[matrix of nodes]{\BODY};
        \pgfmathtruncatemacro\endmtx{\thelines-1}

        \path[timeline color st]
        ($(vtimeline\thevtml-1-1.north east)!0.5!(vtimeline\thevtml-1-2.north west)$)--
        ($(vtimeline\thevtml-\endmtx-1.south east)!0.5!(vtimeline\thevtml-\endmtx-2.south west)$);

        \foreach \x in {1,...,\endmtx}{
                \node[circle,timeline color st, inner sep=0.15pt, draw=white, thick]
                (vtimeline\thevtml-c-\x) at
                ($(vtimeline\thevtml-\x-1.east)!0.5!(vtimeline\thevtml-\x-2.west)$){};
                \draw[timeline color st](vtimeline\thevtml-c-\x.west)--++(-3pt,0);
            }

        \ifvtimelinetitle%
            \draw[timeline color st]([yshift=\lineoffset]vtimeline\thevtml.north west)--
            ([yshift=\lineoffset]vtimeline\thevtml.north east);

            \node[anchor=west,yshift=16pt,font=\large]
            at (vtimeline\thevtml-1-1.north west)
            {\textsc{Timeline \thevtml}: \textit{\vtimelinetitle}};
        \else%
            \relax%
        \fi%

        \ifvtimebottomline%
            \draw[timeline color st]([yshift=-\lineoffset]vtimeline\thevtml.south west)--
            ([yshift=-\lineoffset]vtimeline\thevtml.south east);
        \else%
            \relax%
        \fi%
    \end{tikzpicture}
}

% \usepackage[amsthm]{newpxtext}
% \usepackage[amsthm]{newpxmath}

% \setsansfont{texgyreheros}[
%     Scale=MatchLowercase,
%     UprightFont=*-regular,
%     BoldFont=*-bold,
%     ItalicFont=*-italic,
%     BoldItalicFont=*-bolditalic,
% ]






% \begin{vtimeline}[timeline color=cyan!80!blue, add bottom line, line offset=2pt, use timeline header,timeline title={Hitos de las redes neuronales artificiales}]
%     1676        & The Chain Rule \cite{leibniz2012early}                                                    \endlr
%     1847        & Augustin-Louis Cauchy \cite{lemarechal2012cauchy}                                         \endlr
%     1943        & Threshold Logic Unit (TLU) \cite{mcculloch1943logical}                                    \endlr
%     1949        & Teoría Hebbiana                                                                           \endlr
%     1958        & Perceptron \cite{rosenblatt1958perceptron}                                                \endlr
%     1959-1960   & Adaline y Madaline \cite{rosenblatt1958perceptron}                                        \endlr
%     1965        & Multilayer Perceptron (MLP) \cite{baum1988capabilities}                                   \endlr
%     1967-1968   & Deep Learning by Stochastic Gradient Descent \cite{karplus19671967}                       \endlr
%     1980`s      & Neuronas Sigmoidales                                                                      \endlr
%     ~           & Feedforward neural network (FNN) \cite{rumelhart1985learning}                             \endlr
%     ~           & Backpropagation (BP) \cite{rosenblatt1962principles,etde_5080493,lecun1985learning}       \endlr
%     1985        & Boltzmann Machine \cite{ACKLEY1985147}                                                    \endlr
%     1987        & Adaptive resonance theory (ART) \cite{grossberg1987competitive}                           \endlr
%     1989        & Convolutional neural networks (CNN) \cite{lecun1989backpropagation}                       \endlr
%     ~           & Recurent neural networks (RNN) \cite{schmidhuber1993habilitation}                         \endlr
%     1990        & Generative Adversarial Networks (GAN) as Game \cite{schmidhuberunsupervised}              \endlr
%     1997        & Long short term memory (LSTM) \cite{Hochreiter1997LongSM, hochreiter1997long}             \endlr
%     2006        & Deep Belief Networks (DBN) \cite{hinton2006fast}                                          \endlr
%     ~           & Restricted Boltzmann Machine \cite{hinton2006reducing}                                    \endlr
%     ~           & Encoder / Decoder (Auto-encoder) \cite{hinton2006reducing}                                \endlr
%     2014        & Generative Adversarial Networks (GAN) Moderns \cite{6294131,goodfellow2014generative}     \endlr
%     2018        & Style Generative Adversarial Networks (Style-GAN) \cite{karras2019stylebased}             \endlr
% \end{vtimeline}





% Comando que se encarga de componer la portada del TFG
\newcommand{\Portada}{ %
    \thispagestyle{empty}
    \ULCornerWallPaper{1}{figures/portada-master.pdf}
    \begin{textblock*}{14cm}(5cm,13cm) 
        \centering
        {\fontsize{32}{38}\selectfont \textbf{\TituloTFG}}
    \end{textblock*}
    \begin{textblock*}{10cm}(8cm,20.6cm)
      {\fontsize{16}{19}\selectfont \Estudiante}
    \end{textblock*}
    \begin{textblock*}{10cm}(7.2cm,21.9cm)
      {\fontsize{16}{19}\selectfont \TutorA}
    \end{textblock*}
    \begin{textblock*}{10cm}(7.2cm,23.2cm)
      {\fontsize{16}{19}\selectfont \Departamento}
    \end{textblock*}
    \begin{textblock*}{10cm}(9.5cm,27cm)
      {\fontsize{16}{19}\selectfont \textbf{\FechaPortada}}
    \end{textblock*}
    ~
    \clearpage
    \ClearWallPaper
    \thispagestyle{empty}
    
    \cleardoublepage

    \thispagestyle{empty}
    \begin{figure}
        \centering
        \includegraphics[width=.4\textwidth]{figures/uja.jpg}
    \end{figure}
    
    \vspace*{4em}
    
    D. \TutorA, tutor del Trabajo Fin de Máster titulado: \textbf{\TituloTFG}, que presenta \Estudiante, autoriza su presentación para defensa y evaluación en la Escuela Politécnica Superior de Jaén.
    
    \vspace*{2em}
    \begin{center}
        Jaén, \Fecha
    \end{center}
    
    \vspace{2em}
    % \hspace{2cm} El estudiante \hspace{6cm} \ifdefempty{\TutorB}{El tutor}{Los tutores}
    
    % % \vspace{5cm}
    \begin{tabularx}{0.95\linewidth}{Y p{1cm} Y}
        \centering
        El estudiante   &             & El tutor    \\    
                        &             &             \\    
                        &             &             \\    
                        &             &             \\    
                        &             &             \\    
                        &             &             \\    
                        &             &             \\    
                        &             &             \\    
        \Estudiante     &             & \TutorA     \\    
    \end{tabularx}
    \clearpage\thispagestyle{empty}
    \onehalfspacing  % Fijamos el interlineado   
}

\makeglossaries


% \addterm
% {sample}% label
% {slovene translation of the term}%
% {plural form of slovene translation of the term}%
% {english term}%
% {plural english term}%
% {slovene description of the term}
% Términos de seguridad clásica
\newglossaryentry{latex}    {name={latex},      description={Is a mark up language specially suited for scientific documents}}

\newglossaryentry{STRIDE}   {name={STRIDE},     description={Son las siglas de las amenazas de \textit{Spoofing}, \textit{Tampering}, \textit{Repudiation}, \textit{Information}, \textit{Denial}, \textit{Elevation} que violan las propiedades de \textbf{Autenticidad}, \textbf{Integridad}, \textbf{No Repudio}, \textbf{Información}, \textbf{Disponibilidad} y \textbf{Autorización} respectivamente}}
\newglossaryentry{MITRE}    {name={MITRE},      description={La abreviatura de (Tácticas, Técnicas y Conocimiento Amplio de Enemigos) es un marco de trabajo para la evaluación de la seguridad en las organizaciones. \href{https://attack.mitre.org/}{Enlace}}}

% Generales
\newglossaryentry{SDLC}     {name={SDLC},   first={Systems Development Life Cycle (SDLC)},           description={Ciclo de vida del desarrollo de sistema}}
\newglossaryentry{SAST}     {name={SAST},   first={Static application security testing (SAST)},      description={Las pruebas de seguridad de aplicaciones estáticas (SAST), también conocidas como Static Application Security Testing, de pruebas de seguridad}}
\newglossaryentry{DAST}     {name={DAST},   first={Dynamic Application Security Testing (DAST)},     description={Las pruebas de seguridad de aplicaciones dinámicas (DAST), también conocidas como Dynamic Application Security Testing, de pruebas de seguridad}}
\newglossaryentry{IAST}     {name={IAST},   first={Interactive Application Security Testing (IAST)}, description={Las pruebas de seguridad de aplicaciones interactivas (IAST), también conocidas como Interactive Application Security Testing, de pruebas de seguridad}}
\newglossaryentry{LS}     {name={LS},   first={Latent Space (LS)}, description={Un espacio matemático de alta dimensionalidad que representa estructuras abstractas de datos y que son mapeadas por un modelo de AI, permite la manipulación y generación de instancias}}

%-> Terminos del libro


% Términos IA general
\newglossaryentry{Fine-tuning}      {name={Fine-tuning},    description={El \textit{Fine-tuning} es un enfoque para transferir el aprendizaje en el que los pesos de un modelo previamente entrenado se entrenan en datos nuevos}}
\newglossaryentry{AGI}      {name={AGI},        first={Artificial general intelligence (AGI)},          description={Inteligencia artificial general}}
\newglossaryentry{AI}       {name={AI},         first={Artifical Intelligence (AI)},                    description={La inteligencia artificial (IA), también conocida por su nombre inglés, Artificial Intelligence (AI), es una tecnología que trata de realizar las tareas y tomar las decisiones empresariales de forma automática y autónoma, aprendiendo de forma continua. \cite{glosario-tic-artificial-intelligence}}}
\newglossaryentry{ML}       {name={ML},         first={Machine Learning (ML)},                          description={El machine learning es la tecnología que permite que un sistema aprenda de forma continua. El sistema recibe un input, un humano reacciona y, así, la próxima vez que el sistema reciba ese input, sabrá cómo actuar sin necesidad de acudir al humano. \cite{glosario-tic-machine-learning}}}
\newglossaryentry{DL}       {name={DL},         first={Deep Learning (DL)},                             description={\textit{Deep learning (DL)}, también conocido como aprendizaje profundo, es un tipo de machine learning que se estructura inspirándose en el cerebro humano y sus redes neuronales. El aprendizaje profundo procesa datos para detectar objetos, reconocer conversaciones, traducir idiomas y tomar decisiones. Al ser un tipo de machine learning, esta tecnología sirve para que la inteligencia artificial aprenda de forma continua. \cite{glosario-tic-deep-learning}}}
\newglossaryentry{KDD}      {name={KDD},        first={Knowledge Discovery in Databases (KDD)},         description={Es el proceso utilizado para extraer de forma eficiente y automática \textbf{información útil} a partir de grandes volumenes de datos}}

\newglossaryentry{LTU}      {name={LTU},        first={Linear Threshold Unit (LTU)},                    description={La unidad de umbral lineal es una neurona artificial muy simple cuya salida es el sumatorio de la entrada total umbralizada. Es decir, una \texttt{LTU} con umbral \texttt{T} calcula la suma ponderada de sus entradas y, a continuación, emite \textbf{0} si esta suma es inferior a \texttt{T} y \textbf{1} si la suma es superior a \texttt{T}. Las \texttt{LTU} constituyen la base de los perceptrones. \cite{mldict}}}
\newglossaryentry{AL}       {name={AL},         first={Active Learning (AL)},                           description={Aprendizaje activo}}
\newglossaryentry{BP-NN}    {name={BP},         first={Backpropagation (BP)},                           description={Método utilizado en redes neuronales para calcular el gradiente y el cálculo de los pesos, es una abreviatura de ``propagación de errores hacia atrás''}}
\newglossaryentry{MLP}      {name={MLP},        first={Multilayer Perceptron (MLP)},                    description={El Perceptron de multiples capas es una red neuronal artificial formada por capas local o totalmente conectadas}}
\newglossaryentry{ART-NN}   {name={ART},        first={Adaptive Resonance Theory (ART)},                description={Modelo de red neuronal que equilibra la adaptación a nueva información con la estabilidad frente a patrones familiares}}
\newglossaryentry{LDA}      {name={LDA},        first={Linear discriminant analysis (LDA)},             description={Es un método de discriminación lineal que se usa para encontrar una combinación lineal de rasgos que caracterizan a dos o más clases}}
\newglossaryentry{ACC}      {name={ACC},        first={Accuracy on a clean test/evaluation set (ACC)},  description={Precisión}}
\newglossaryentry{AUC}      {name={AUC},        first={Area Under the (ROC) Curve (AUC)},               description={Área bajo la curva (ROC)}}
\newglossaryentry{PAUC}     {name={PAUC},       first={Partial (ROC) Area Under the Curve (PAUC)},      description={Área parcial bajo la curva (ROC)}}

\newglossaryentry{PCA}      {name={PCA},    first={Principal Component Analysis (PCA)}, description={Es una técnica para describir un conjunto de datos de nuevas variables no correlacionadas, se utiliza principalmente en análisis exploratorio de datos}}

%-> Terminos del libro


\newglossaryentry{VAE}      {name={VAE},        first={Variational Autoencoder (VAE)},                                      description={Un codificador automático variacional es un tipo de modelo generativo basado en probabilidad.}}
% Tipos de redes
\newglossaryentry{RNN}      {name={RNN},        first={Recurrent Neural Network (RNN)},                                     description={Es una clase de redes neuronales en la que las conexiones entre nodos forma un grafo dirigido a lo largo de una secuencia iterativa temporal. A diferencia de las FNN, las RNN puede usar sus pesos internos para procesar secuencias de entrada}}
\newglossaryentry{GNN}      {name={GNN},        first={Graph Neural Network (GNN)},                                         description={Es una clase de redes neuronales especializada en el procesamiento de datos que se puedan representar como gráficos}}
\newglossaryentry{LSTM}     {name={LSTM},       first={Long Short-Term Memory (LSTM)},                                      description={Memoria larga a corto plazo}}
\newglossaryentry{ResNet-n} {name={ResNet-n},   first={Residual Neural Network architecture with $n$ layers (ResNet-n)},    description={Arquitectura de red neuronal residual con $n$ capas}}
\newglossaryentry{LeNet-n}  {name={LeNet-n},    first={Learnable Neural Network architecture with $n$ layers (LeNet-n)},    description={Arquitectura de red neuronal de aprendizaje con $n$ capas}}
%-> Terminos del libro
\newglossaryentry{Inpainting}     {name={Inpainting}, description={Técnica de generación de imágenes de AI utilizada para rellenar las partes que faltan en una imagen}}


% Tareas
\newglossaryentry{Pix2Pix}  {name={Pix2Pix},    first={Pixel to Pixel (Pix2Pix)},                   description={\textit{Pixel to Pixel (Pix2Pix)}, es la abreviatura para tareas de traducción de imágenes}}


% Otras
\newglossaryentry{U-Net-GAN}    {name={U-Net-GAN},  first={A U-Net Based Discriminator for GAN (U-Net-GAN)},        description={}}
\newglossaryentry{U-Net}        {name={U-Net},      first={Convolutional Networks (U-Net)},                         description={U-Net son las siglas de ``red en forma de U''. Se trata de una arquitectura de codificador-decodificador con conexiones de salto entre las capas reflejadas de las pilas de codificadores y decodificadores.}}

% Métodos

\newglossaryentry{MDE}      {name={MDE},        first={Monocular Depth Estimation (MDE)},                           description={Es la tarea de estimar la profundidad a partir de un solo cuadro}}
\newglossaryentry{SPEED}    {name={SPEED},      first={Separable Pyramidal Pooling EncodEr-Decoder for Real-Time Monocular Depth Estimation on Low-Resource Settings (SPEED)},  description={}}


% Estadistica
\newglossaryentry{OOD}  {name={OOD},        first={Out-Of-Distribution (OOD)},                  description={Fuera de distribución}}
\newglossaryentry{OODD} {name={OODD},       first={Out-Of-Distribution Detection (OODD)},       description={Detección fuera de distribución}}
\newglossaryentry{pdf}  {name={pdf},        first={probability density function (pdf)},         description={función de densidad de probabilidad}}
\newglossaryentry{pmf}  {name={pmf},        first={probability mass function (pmf)},            description={función de masa de probabilidad}}
\newglossaryentry{LR}   {name={LR},         first={Logistic Regression (LR)},                   description={Regresión logística}}
\newglossaryentry{BIC}  {name={BIC},        first={Bayesian Information Criterion (BIC)},       description={Criterio de información bayesiano}}
\newglossaryentry{LEM}  {name={LEM},        first={Local Error Maximizer (LEM)},                description={Maximizador de errores locales}}
\newglossaryentry{MAD}  {name={MAD},        first={Median Absolute Deviation (MAD)},            description={Desviación absoluta mediana}}
\newglossaryentry{MAE}  {name={MAE},        first={Mean Absolute Error (MAE)},                  description={Error absoluto medio}}
\newglossaryentry{MAP}  {name={MAP},        first={Maximum a posteriori (MAP)},                 description={Máximo a posteriori}}
\newglossaryentry{MLE}  {name={MLE},        first={Maximum Likelihood Estimation (MLE)},        description={Estimación de máxima verosimilitud}}
\newglossaryentry{MM}   {name={MM},         first={Mixture Model (MM)},                         description={Modelo de mezcla}}
\newglossaryentry{MSE}  {name={MSE},        first={Mean-Squared Error (MSE)},                   description={Error medio cuadrático}}
\newglossaryentry{NB}   {name={NB},         first={Native Bayes (NB)},                          description={Bayes ingenuo}}
%-> Terminos del libro


% Ingenieria inversa
% \newglossaryentry{RE}       {name={RE (Reverse-Engineering)},                           description={Ingeniería inversa}}
% \newglossaryentry{RE-AP}    {name={RE-AP (Reverse-Engineering Additive Perturbation)},  description={Perturbación aditiva de ingeniería inversa}}
% \newglossaryentry{RE-PR}    {name={RE-PR (Reverse-Engineering Patch Replacement)},      description={Reemplazo de parches de ingeniería inversa}}
% \newglossaryentry{REA}      {name={REA (Reverse-Engineering Attack)},                   description={Ataques de ingeniería inversa}}
% \newglossaryentry{RED}      {name={RED (Reverse-Engineering Defense)},                  description={Defensas frente ingeniería inversa}}
\newglossaryentry{RE}       {name={RE},     first={Reverse-Engineering (RE)},                           description={\textit{Reverse-Engineering (RE)}. Ingeniería inversa}}
\newglossaryentry{RE-AP}    {name={RE-AP},  first={Reverse-Engineering Additive Perturbation (RE-AP)},  description={\textit{Reverse-Engineering Additive Perturbation (RE-AP)}. Perturbación aditiva de ingeniería inversa}}
\newglossaryentry{RE-PR}    {name={RE-PR},  first={Reverse-Engineering Patch Replacement (RE-PR)},      description={\textit{Reverse-Engineering Patch Replacement (RE-PR)}. Reemplazo de parches de ingeniería inversa}}
\newglossaryentry{REA}      {name={REA},    first={Reverse-Engineering Attack (REA)},                   description={\textit{Reverse-Engineering Attack (REA)}. Ataques de ingeniería inversa}}
\newglossaryentry{RED}      {name={RED},    first={Reverse-Engineering Defense (RED)},                  description={\textit{Reverse-Engineering Defense (RED)}. Defensas frente ingeniería inversa}}

%-> Terminos del libro


% Análisis
\newglossaryentry{AD} {name={AD}, first={Anomaly Detection (AD)}, description={Detección de anomalías}}

%-> Terminos del libro
\newglossaryentry{ADA} {name={ADA}, first={Anomaly Detection of TTE Attacks (ADA)}, description={Detección de anomalías en ataques de tipo TTE}}

% Ataques
% \newglossaryentry{ASR}  {name={ASR (Attack Success Rate)},          description={Tasa de éxito del ataque}}
% \newglossaryentry{BA}   {name={BA (Backdoor Attack (Trojan))},      description={Ataque de puerta trasera (troyano)}}
% \newglossaryentry{DP}   {name={DP (Data Poisoning (attack))},       description={Envenanimiento de los datos}}
% \newglossaryentry{BP}   {name={BP (Backdoor Pattern)},              description={Patrones de puerta trasera}}
% \newglossaryentry{TTE}  {name={TTE (Test-Time Evasion (attack))},   description={Evasión en el tiempo de prueba}}
\newglossaryentry{ASR}              {name={ASR},    first={Attack Success Rate (ASR)},      description={\textit{Attack Success Rate (ASR)}. Tasa de éxito del ataque}}
\newglossaryentry{BA}               {name={BA},     first={Backdoor Attack (Trojan) (BA)},  description={\textit{Backdoor Attack (BA)}. Ataque de puerta trasera (troyano)}}
\newglossaryentry{DP}               {name={DP},     first={Data Poisoning (DP)},            description={\textit{Data Poisoning (DP)}. Envenenamiento de los datos}}
\newglossaryentry{BP-backdoor}      {name={BP},     first={Backdoor Pattern (BP)},          description={\textit{Backdoor Pattern (BP)}. Ataque por patrones de puerta trasera}}
\newglossaryentry{TTE}              {name={TTE},    first={Test-Time Evasion (TTE)},        description={\textit{Test-Time Evasion (TTE)}. Evasión en el tiempo de prueba}}
\newglossaryentry{DGM}              {name={DGM},    first={Deep Generative Models (DGMs)},  description={\textit{Test-Time Evasion (TTE)}. Evasión en el tiempo de prueba}}


% Envenenamiento -> puerta trasera
\newglossaryentry{BadNet-OGA}{name={OGA}, first={Object Generation Attack (OGA)}, description={}}
\newglossaryentry{BadNet-RMA}{name={RMA}, first={Regional Misclassification Attack (RMA)}, description={}}
\newglossaryentry{BadNet-GMA}{name={GMA}, first={Global Misclassification Attack (GMA)}, description={}}
\newglossaryentry{BadNet-ODA}{name={ODA}, first={Object Disappearance Attack (ODA)}, description={}}
%-> Terminos del libro


% Defensas
%-> Terminos del libro
\newglossaryentry{PT}  {name={PT}, first={Post-training (PT)},   description={Posterior al entrenamiento}}
\newglossaryentry{FGM}{name={FGM}, first={Fast Gradient Method (FGM)},   description={}}
\newglossaryentry{CLEVER}{name={CLEVER}, first={Cross Lipschitz Extreme Value for nEtwork Robustness (CLEVER)},   description={}}


% Términos de internet
\newglossaryentry{LotL}  {name={LotL}, first={living-off-the-land (LotL)},   description={Ataques a través de programas confiables.}}

% \newglossaryentry{AdvML}  {name={AdvML},      description={Aprendizaje automático adversarial}}
% \newglossaryentry{BIM}    {name={BIM},        description={Método iterativo básico}}
% \newglossaryentry{CNN}    {name={CNN},        description={Redes neuronales convolucionales}}
% \newglossaryentry{DDoS}   {name={DDoS},       description={Denegación de servicio distribuida}}
% \newglossaryentry{DT}     {name={DT},         description={Árbol de decisión}}
% \newglossaryentry{FFNN}   {name={FFNN},       description={Red neuronal directa}}
% \newglossaryentry{FGSM}   {name={FGSM},       description={Método de signo de gradiente rápido}}
% \newglossaryentry{FNR}    {name={FNR},        description={Tasa de falsos negativos}}
% \newglossaryentry{GAN}    {name={GAN},        description={Red Generativa Adversarial}}
% \newglossaryentry{GB}     {name={GB},         description={Refuerzo de gradiente}}
% \newglossaryentry{GMM}    {name={GMM},        description={Modelo de mezcla gaussiana}}
% \newglossaryentry{HIDS}   {name={HIDS},       description={IDS en casa}}
% \newglossaryentry{HSJ}    {name={HSJ},        description={Hop Skip Jump}}
\newglossaryentry{IDS}    {name={IDS}, first={Intrusion Detection System (IDS)},          description={Sistema de detección de intrusos}}
% \newglossaryentry{NIDS}   {name={NIDS},       description={IDS en red}}
% \newglossaryentry{IoT}    {name={IoT},        description={Internet de los objetos}}
% \newglossaryentry{JSMA}   {name={JSMA},       description={Ataque al mapa de saliencia basado en jacobianos}}
% \newglossaryentry{KNN}    {name={KNN},        description={Vecinos más próximos K}}
% \newglossaryentry{PGD}    {name={PGD},        description={Descenso gradual proyectado}}
% \newglossaryentry{SOA}    {name={SOA},        description={Estado de la técnica}}
% \newglossaryentry{VPC}    {name={VPC},        description={Clasificador de vectores de apoyo}}
% \newglossaryentry{SVM}    {name={SVM},        description={Máquina de vectores soporte}}
% \newglossaryentry{TAC}    {name={TAC},        description={Precisión total de la predicción}}
% \newglossaryentry{PAU}    {name={PAU},        description={Perturbaciones Adversariales Universales}}

\newacronym{nn}{NN}{Neural Network}
\newacronym{ANN}{ANN}{Artificial Neural Network}
\newacronym{dnn}{DNN}{Deep Neural Network}
\newacronym{CNN}{CNN}{Convolutional Neural Network}
\newacronym{mlp}{MLP}{Multilayer Perceptron}
\newacronym{fnn}{FNN}{Feedforward Neural Network}
\newacronym{rnn}{RNN}{Recurrent Neural Network}
\newacronym{gnn}{GNN}{Graph Neural Network}
\newacronym{dbn}{DBN}{Deep Belief Networks}
\newacronym{BP}{BP}{Backpropagation}
\newacronym{ai}{AI}{Artifical Intelligence}
\newacronym{xai}{XAI}{eXplicable AI}
\newacronym{gan}{GAN}{Generative Adversarial Network}

\newacronym{lstm}{LSTM}{Long Short-Term Memory Network}
\newacronym{art-nn}{ART}{Adaptive Resonance Theory}
\newacronym{RL}{RL}{Reinforcer learn}





% Tipos de GANS
\newacronym{GAN}            {GAN}         {Generative Adversarial Network}
\newacronym{SGAN}           {SGAN}        {Semi supervised GAN}
\newacronym{cGAN}           {cGAN}        {Conditional GAN}
\newacronym{CoGAN}          {CoGAN}       {Couple GAN}
\newacronym{CycleGAN}       {CycleGAN}    {Cycle GAN}
\newacronym{DCGAN}          {DCGAN}       {Deep Convolutional GAN}
\newacronym{LSGAN}          {LSGAN}       {Least Square GAN}
\newacronym{ProGAN}         {ProGAN}      {Progressive GAN}
\newacronym{PGGAN}          {PGGAN}       {Progressive Growing GAN}
\newacronym{SRGAN}          {SRGAN}       {Super Resolution GAN}
\newacronym{ESRGAN}         {ESRGAN}      {Enhanced Super Resolution GAN}
\newacronym{StyleGAN}       {StyleGAN}    {Style GAN}
\newacronym{SAGAN}          {SAGAN}       {Self-Attention GAN}
\newacronym{DRAGAN}         {DRAGAN}      {Discriminator Regularization Auxiliary GAN}
\newacronym{WGAN}           {WGAN}        {Wasserstein GAN}
\newacronym{WGAN-GP}        {WGAN-GP}     {Wasserstein GAN Gradient Penalty}
\newacronym{WGAN-div}       {WGAN-div}    {Wasserstein Divergence GAN}
\newacronym{SSGAN}          {SSGAN}       {Secure Steganography GAN}
\newacronym{InfoGAN}        {InfoGAN}     {Information Maximizing GAN}
\newacronym{InterfaceGAN}   {InterfaceGAN}{Interface GAN}
\newacronym{PatchGAN}       {PatchGAN}    {Patch GAN}
\newacronym{ConvNET}        {ConvNET}    {Convolutional Neural Networks}

% Métodos
\newacronym{FGSM}           {FGSM}       {Fast Gradient Signed Method}
\newacronym{FGNSM}          {FGNSM}      {Fast Gradient Non-Sign Method}
\newacronym{DiTs}           {DiTs}       {Diffusion Transformers}
% Otros términos
\newacronym{PSNR}           {PSNR}       {Peak Signal-to-Noise Ratio}





% a.s.:               almost surely (with probability one)
% ET:                 Expected Transferability

% CDF or cdf:         Cumulative Distribution Function
% GMM: Gaussian Mixture Model
% HC: High Confidence
% i.i.d.: independent and identically distributed
% JSD: Jensen Shannon Divergence
% KL: KulIback LeibIer divergence
% K NN: K Nearest Neighbors
% LC: Low Confidence
% PMM: Parsimonious Mixture Modeling
% PCA: Principal Component Analysis




%%% Términos
% -CS:                 Cosine Similarity
% -AUC:                Area Under the (ROC) Curve
% -PAUC: partial (ROC) Area Under the Curve
% -ACC:                Accuracy (on a clean test/evaluation set)
% -Al:                 Artificial Intelligence (often synonymous with a DNN)
% -AL:                 Active Learning
% -ML: Machine Learning
% -RL: Reinforcement Learning
% -ROC: Receiver Operating Characteristic

% PT: Post-Training
% FPR: False Positive Rate (fraction or percentage)
% TPR: True Positive Rate (fraction or percentage)

% SGD: Stochastic Gradient Descent
% SVM: Support Vector Machine
% TSC: Training Set Cleansing

% SIA: Source-class Inference Accuracy
% SVD: Singular Value Decomposition
% Al: eXplainabIe Al
% WB: White Box


%%% Tipos de redes
% -DNN:                Deep Neural Network
% -CNN:                Convolutional Neural Network
% -NN: Neural Network
% -GAN: Generative Adversarial Network
% -LSTM: Long Short-Term Memory (a recurrent NN)
% -ResNet-n: Residual Neural Network architecture with n layers
% -LeNet-n: Learnable Neural Network architecture with n layers


%%% Estadistica
% -BIC:                Bayesian Information Criterion
% -LR: Logistic Regression
% -MAD:    Median Absolute Deviation
% -MAE:    Mean Absolute Error
% -LEM: Local Error Maximizer
% -MAP:    Maximum a posteriori
% -MLE:    Maximum Likelihood Estimation
% -MM:     Mixture Model (or Maximum Margin in Chapter 9)
% -MSE:    Mean-Squared Error
% -NB:     Naive Bayes
% -OOD: Out-Of-Distribution
% -OODD: Out-Of-Distribution Detection
% -pdf: probability density function
% -pmf: probability mass function


%%% Ingenieria inversa
% -RE: Reverse-Engineering
% -RE-AP: Reverse-Engineering Additive Perturbation
% -RE-PR: Reverse-Engineering Patch Replacement
% -REA: Reverse-Engineering Attack
% -RED: Reverse-Engineering Defense


%%% Análisis
% -AD:                 Anomaly Detection (short name for 1-PT-RED in Chapter 6)


%%% Ataques
% -ASR:                Attack Success Rate
% -BP:                 Backdoor Pattern
% -DP:                 Data Poisoning (attack)
% -BA:                 Backdoor Attack (Trojan)
% TTE: Test-Time Evasion (attack), that is, adversarial input


%%% Defensas





\begin{document}
% \setemojifont{Twemoji Mozilla}

\Portada~


\pagenumbering{roman}  % Numeración romana para los agradecimientos, dedicatoria y tablas de contenidos

\input{0_Agradecimientos.tex} % Editar este archivo para introducir los agradecimientos/dedicatoria

\cleardoublepage
\tableofcontents  % Tabla de contenidos

% ===== Comentar o eliminar las líneas de los índices que no deseen incluirse al inicio de la memoria

% \clearpage\thispagestyle{empty}\cleardoublepage
% \listoffigures		% Índice de figuras

% \clearpage\thispagestyle{empty}\cleardoublepage
% \listoftables 		% Índice de tablas

% \clearpage\thispagestyle{empty}\cleardoublepage
% \listofalgorithms

% \clearpage\thispagestyle{empty}\cleardoublepage
% \lstlistoflistings		% Índice de listados

% \clearpage\thispagestyle{empty}\cleardoublepage
\pagenumbering{arabic} % Numeración arábiga para el resto del documento

% ===== Archivos LaTeX con los distintos capítulos que componen la memoria

% \input{0__Tutorial} 
% \clearpage\thispagestyle{empty}\cleardoublepage

% \chapter*{Pruebas}

\section*{Emojis}

Añadimos emoji con lualatex \emoji{joy}

\section*{Def}

\subsection*{Acronym}

Añadimos acronimo con \texttt{makeglossaries} \acrshort{gcd}

\subsection*{Glossaries}

Añadimos glosario con \texttt{makeglossaries} \gls{ACC}

\section*{Matemáticas}

% \section{Simple equations}
A simple equation:
\[
    f(x)=(x+a)(x+b)
\]
An equation with text:
\begin{equation}
    50 \text{ apples} \times 100 \text{ apples} =
    \textbf{lots of apples}
\end{equation}
One including subscripts and superscripts:
\[ k_{n+1} = n^2 + k_n^2 - k_{n-1} \]

\section{Greek Letters}
\[ \alpha, \beta, \gamma, \Gamma, \pi, \Pi, \phi, \varphi, \mu, \Phi, \xi, \zeta \]
\[ \cos(2\theta\phi) = \cos^2 \theta\phi - \sin^2 \theta\phi \]

\section{Delimiters}
There are many types of delimiters one can use:
\[ ( a ), [ b ], \{ c \}, | d |, \| e \|,
    \langle f \rangle, \lfloor g \rfloor,
    \lceil h \rceil, \ulcorner i \urcorner \]
See how the delimiters are of reasonable size in these examples
\[
    \left(a+b\right)\left[1-\frac{b}{a+b}\right]=a\,,
\]
\[
    \sqrt{|xy|}\leq\left|\frac{x+y}{2}\right|,
\]
even when there is no matching delimiter
\[
    \int_a^bu\frac{d^2v}{dx^2}\,dx
    =\left.u\frac{dv}{dx}\right|_a^b
    -\int_a^b\frac{du}{dx}\frac{dv}{dx}\,dx.
\]
whereas vector problems often lead to statements such as
\[
    u=\frac{-y}{x^2+y^2}\,,\quad
    v=\frac{x}{x^2+y^2}\,,\quad\text{and}\quad
    w=0\,.
\]

\section{Multiple Fractions}
Typesetting continued fractions is easy:
\[
    x = a_0 + \frac{1}{a_1 + \frac{1}{a_2 + \frac{1}{a_3 + a_4}}}
\]
However, as the fractions continue, they get smaller. If you want to keep the size consistent, use the display style; e.g.
\[
    x = a_0 + \frac{1}{\displaystyle a_1
        + \frac{1}{\displaystyle a_2
            + \frac{1}{\displaystyle a_3 + a_4}}}
\]

\section{Arrays}
Arrays of mathematics are typeset using one of the matrix environments as
in
\[
    \begin{bmatrix}
        1 & x & 0  \\
        0 & 1 & -1
    \end{bmatrix}\begin{bmatrix}
        1 \\
        y \\
        1
    \end{bmatrix}
    =\begin{bmatrix}
        1+xy \\
        y-1
    \end{bmatrix}.
\]
\[ \begin{pmatrix}
        2 & 3 & 4  \\
        5 & 6 & 7  \\
        8 & 9 & 10\end{pmatrix} v = 0 \]
Case statements use cases:
\[
    |x|=\begin{cases}
        x,  & \text{if }x\geq 0\,, \\
        -x, & \text{if }x< 0\,.
    \end{cases}
\]
Many arrays have lots of dots all over the place as in
\[
    \begin{matrix}
        -2     & 1      & 0      & 0      & \cdots & 0      \\
        1      & -2     & 1      & 0      & \cdots & 0      \\
        0      & 1      & -2     & 1      & \cdots & 0      \\
        0      & 0      & 1      & -2     & \ddots & \vdots \\
        \vdots & \vdots & \vdots & \ddots & \ddots & 1      \\
        0      & 0      & 0      & \cdots & 1      & -2
    \end{matrix}
\]

\section{Greek Letters}
\[ \alpha,  \beta,  \gamma, \Gamma, \pi, \Pi, \phi, \varphi, \mu, \Phi, \xi, \zeta \]
\[ \cos(2\theta\phi) = \cos^2 \theta\phi - \sin^2 \theta\phi \]

\section{Delimiters}
\[ ( a ), [ b ], \{ c \}, | d |, \| e \|,
    \langle f \rangle, \lfloor g \rfloor,
    \lceil h \rceil, \ulcorner i \urcorner
\]

\section{Accents}
Mathematical accents are performed by a short command with one
argument, such as
\[
    \tilde f(\omega)=\frac{1}{2\pi}
    \int_{-\infty}^\infty f(x)e^{-i\omega x}\,dx\,,
\]
or
\[
    \dot{\vec \omega}=\vec r\times\vec I\,.
\]

\section{Multiline equations and aligned environments}

New lines (\textbackslash \textbackslash) do not work in equation environments. To achieve alignment of equations, use the aligned  package to produce multiline aligned math, such as:

\begin{center}
    \begin{align}
        F = & F_{x} \in  F_{c} : (|S| > |C|)                        \\
            & \cap (\mathrm{minPixels}  < |S| < \mathrm{maxPixels}) \\
            & \cap (|S_{\mathrm{conected}}| > |S| - \epsilon)
    \end{align}
\end{center}

and also:
\newline
\begin{center}
    \begin{align}
        A_0 = & \frac{1}{(\alpha+t_x)^{r+s+x}}{}_2 F_1\left( r+s+x,x+1;r+s+x+1;\frac{\alpha-\beta}{\alpha + t_x} \right)      \\
              & \quad - \frac{1}{(\alpha+T)^{r+s+x}}{}_2 F_1\left( r+s+x,x+1;r+s+x+1;\frac{\alpha-\beta}{\alpha + T} \right),
    \end{align}
\end{center}

\textbf{Note}: the above multiline equations have math mode defined per line, not globally at the equation level.

\section{Theorems and sets}
\newtheorem{theorem}{Theorem}
\newtheorem{corollary}[theorem]{Corollary}
\newtheorem{lemma}[theorem]{Lemma}
\newtheorem{definition}[theorem]{Definition}
\begin{theorem}
    For any non-negative integer ${n}$, we have
    $${(1+x)^{n} = \sum_{i=0}^{n} {n \choose i} x^i}$$
\end{theorem}
The Taylor series expansion for the function ${e^x}$ is given by
\begin{equation}
    e^x = 1 + x + \frac{x^2}{2} + \frac{x^3}{6} + \cdots = \sum_{n\geq 0} \frac{x^n}{n!}
\end{equation}
\[ \forall x \in X, \quad \exists y \leq \epsilon \]
\[ \frac{n!}{k!(n-k)!} = \binom{n}{k} \]
\begin{theorem}
    For any sets ${A}$, ${B}$ and ${C}$, we have
    $${(A\cup B)-(C-A) = A \cup (B-C)}$$
\end{theorem}

% \clearpage\thispagestyle{empty}\cleardoublepage

% \chapter*{Notación}
\addcontentsline{toc}{chapter}{\protect\numberline{}Notación}

% TODO: https://nthu-datalab.github.io/ml/slides/Notation.pdf

\begin{table}[ht!]
    \begin{center}
        \begin{tabularx}{\textwidth}{|r|X|}
            \hline
            \textbf{Símbolo}                                                  & \textbf{Definición}                                                                                                        \\
            \hline
            $ \mathbb{Z} $                                                    & Números enteros. \newline i.e. \scriptsize{${ \mathbb{Z}  = \left\{ \ldots, -2, -1, 0, +1, +2,~\ldots \right\} }$}         \\
            $ \mathbb{Z}^{+} $                                                & Números enteros positivos. \newline \scriptsize{i.e. ${ \mathbb{Z}^{+}  = \left\{ 0, +1, +2,~\ldots \right\} }$}           \\
            $ \mathbb{Z}^{++} $                                               & Números enteros positivos sin el cero. \newline i.e. \scriptsize{${ \mathbb{Z}^{++}  = \left\{ +1, +2,~\ldots \right\} }$} \\
            $ \mathbb{N} $                                                    & Números naturales                                                                                                          \\
            $ \mathbb{R} $                                                    & Números reales                                                                                                             \\
            $ \mathbb{C} $                                                    & Números complejos                                                                                                          \\
            $ R^{n} $                                                         & Espacio vectorial $n$-dimensional de números reales                                                                        \\

            % $\mathcal{A} \backslash \mathcal{B}  $ & $\mathcal{A}$ sin $\mathcal{B}$, el conjunto de elementos de $\mathcal{A}$, pero no en $\mathcal{B} $                       \\
            % $\exists{x}$                           & Existe $x$                                                                                                                  \\
            % $\forall{x}$                           & Para todo $x$                                                                                                               \\
            % $ a := b $                             & $a$ se define como $b$                                                                                                      \\
            % $ a \propto  b $                       & $a$ proporcional a $b$                                                                                                      \\
            $ \epsilon $                                                      & Para cantidades, artbitrariamente pequeñas                                                                                 \\

            \hline % Notación de vectores y matrices
            $ \mathVector{a} = [a_{i}]_{i=1, ~\ldots, n} $                    & Vectores definidos en minuscula, negrita y cursiva                                                                         \\
            $ \mathMatrix{A} = [a_{i,j}]_{i=1, ~\ldots, n, j=1, ~\ldots, m} $ & Matrices definidas en mayúscula, negrita y cursiva                                                                         \\
            $ \mathTensor{A} $                                                & Tensores definidos en mayúscula, negrita y en estilo sans serif                                                            \\
            \hline
            $ I_{n} $                                                         & Matriz identidad de tamaño $ n \times n $                                                                                  \\
            $ 0_{n,m} $                                                       & Matriz de ceros de tamaño $ n \times m $                                                                                   \\
            $ \underline{0}_{n} $                                             & Vector de ceros de tamaño $ n $                                                                                            \\
            $ 1_{n,m} $                                                       & Matriz de unos de tamaño $ n \times m $                                                                                    \\
            $ \underline{1}_{n} $                                             & Vector de unos de tamaño $n $                                                                                              \\
            $ e_{i} $                                                         & Vector estandar o vector canonico.\newline i.e. \scriptsize{${ v_{x} = (1,0,0), v_{y} = (0,1,0), v_{z} = (0,0,1)} $}       \\
            \hline
            $ \mathbf{A}^{\dagger} $                                          & Traspuesta de la matriz $\mathbf{A}$                                                                                       \\
            $ \text{rk}\left(\mathbf{A}\right) $                              & Rango de la matriz $\mathbf{A}$                                                                                            \\
            $ \text{tr}\left(\mathbf{A}\right) $                              & Traza de la matriz $\mathbf{A}$                                                                                            \\
            $ \text{det}\left(\mathbf{A}\right) $                             & Determinante de la matriz $\mathbf{A}$                                                                                     \\
            $ \text{Im}\left(\Phi\right) $                                    & Imagen del mapeo lineal $\Phi$                                                                                             \\
            $ \text{ker}\left(\Phi\right) $                                   & Núcleo (espacio nulo) de un mapeo lineal $\Phi$                                                                            \\
            $ \text{span}\left[\mathbf{b}_{1}\right] $                        & Sistema generador de $\mathbf{b}_{1}$                                                                                      \\


            \hline
        \end{tabularx}
        \caption{Notación Parte 1}
        \label{tab:nnotation-part-1}
    \end{center}
\end{table}


\begin{table}[H]
    \begin{center}
        \begin{tabularx}{\textwidth}{|r|X|}
            \hline
            \textbf{Símbolo}             & \textbf{Definición}                                                      \\
            \hline
            % Indices de vectores y matrices
            ${ a_{i} }$                  & Elemento $i$ del vector $\mathit{\mathbf{a}}$, empezando el índice por 1 \\
            ${ \mathMatrix{A}_{i,j} }$   & Elemento ($i,j$) de la matriz $\mathbf{A}$                               \\
            ${ \mathMatrix{A}_{i,:} }$   & Fila $i$ de la matriz $\mathbf{A}$                                       \\
            ${ \mathMatrix{A}_{:,j} }$   & Columna $j$ de la matriz $\mathbf{A}$                                    \\
            ${ \mathTensor{A}_{i,j,k} }$ & Elemento ($i,j,k$) 3D del tensor $\boldsymbol{\mathbf{A}}$               \\
            ${ \mathTensor{A}_{:,:,k} }$ & Desplazamiento 2D del Tensor 3D                                          \\
            \hline
        \end{tabularx}
        \caption{Notación Parte 1.2}
        \label{tab:nnotation-part-1.2}
    \end{center}
\end{table}



\begin{table}[H]
    \begin{center}
        \begin{tabularx}{\textwidth}{|r|X|}
            \hline
            \textbf{Símbolo}                                                                                                     & \textbf{Definición}                                                                         \\
            \hline
            % Notación matemática de operaciones con vectores y matrices
            $ {\lvert \cdot \rvert} $                                                                                            & Determinante o valor absoluto                                                               \\
            $ {\lVert x \rVert}_{p} $                                                                                            & Norma $L^{p}$ de $x$                                                                        \\
            $ {\lVert x \rVert} $                                                                                                & Norma $L^{2}$ de $x$                                                                        \\
            $ {\Vert \underline{x} - \underline{y} \Vert}_{q} $                                                                  & Distancia en la misma dimensión entre $\underline{x}$ e $\underline{y}$                     \\
            $ \underline{x} \odot \underline{m} $                                                                                & Operación por elementos de los vectores o matrices                                          \\
            $ \left\langle \underline{z}, \underline{y} \right\rangle = \underline{z}' \underline{y} = \sum_{j=1}^{N}{z_j y_j} $ & Producto escalar de vectores por columnas $\underline{z}, \underline{y} \in \mathbb{R}^{N}$ \\
            $ {x_{1}, x_{2}, \ldots ,x_{n}} $                                                                                    & Es el conjunto con $n$ elementos                                                            \\
            \hline
        \end{tabularx}
        \caption{Notación Parte 2}
        \label{tab:nnotation-part-2}
    \end{center}
\end{table}


\begin{table}[H]
    \begin{center}
        \begin{tabularx}{\textwidth}{|r|X|}
            \hline
            \textbf{Símbolo}                                                                                                                                                                                                                  & \textbf{Definición}                                                                                   \\
            \hline
            % Notación matemática de gradientes, funciones a minimizar, operador
            % $ g \circ f $                          & Función composiciones (g después de f)                                                                                        \\
            $ f: \mathbb{A} \rightarrow \mathbb{B} $                                                                                                                                                                                          & Una función $f$ con dominio $\mathbb{A}$ y rango $\mathbb{B}$                                         \\
            $ \nabla f = \left( \frac{\partial f}{\partial x}, \frac{\partial f}{\partial y}, \frac{\partial f}{\partial z} \right) $                                                                                                         & Gradiente, es un vector que indica la dirección de mayor pendiente de una superficie en un punto dado \\
            $ \nabla f(\mathbf{a}) \in \mathbb{R}^{n} $                                                                                                                                                                                       & Gradiente de la función $f: \mathbb{R}^{n} \rightarrow \mathbb{R}$ con entrada $\mathbf{a}$           \\
            $ \nabla \cdot f =  \frac{\partial f}{\partial x} + \frac{\partial f}{\partial y} + \frac{\partial f}{\partial z} $                                                                                                               & Divergencia                                                                                           \\
            $ \nabla \times f =  \left( \frac{\partial f}{\partial y} - \frac{\partial f}{\partial z}, \frac{\partial f}{\partial z} - \frac{\partial f}{\partial x}, \frac{\partial f}{\partial x} - \frac{\partial f}{\partial y}  \right)$ & Rotación                                                                                              \\
            $ f_{*} = \text{min}_x~f(x) $                                                                                                                                                                                                     & El valor de función más pequeño de $f$                                                                \\
            $ {x}_x \in \text{arg min}_{x}~f(x) $                                                                                                                                                                                             & El valor $x_{*}$ (conjunto de valores) que minimiza $f$                                               \\
            \hline
        \end{tabularx}
        \caption{Notación Parte 3}
        \label{tab:nnotation-part-3}
    \end{center}
\end{table}


\begin{table}[H]
    \begin{center}
        \begin{tabularx}{\textwidth}{|r|X|}
            \hline
            \textbf{Símbolo}                           & \textbf{Definición}                                                                                                                                                             \\
            \hline % Notación de ANN (redes neuronales, conjunto de datos, entrada, salida, clases, etc)
            $ \mathcal{X} $                            & Conjunto de datos, donde $\mathcal{X} \in \mathbb{R}^{N}$                                                                                                                       \\
            $ N $                                      & Dimensión del espacio muestral de entrada $\mathcal{X}$                                                                                                                         \\
            $ \mathcal{D} $                            & Subconjunto para el entrenamiento, donde $\mathcal{D} \subset \mathcal{X}$                                                                                                      \\
            $ T = \lvert \mathcal{X} \rvert < \infty $ & Número finito de instancias del conjunto de datos $\mathcal{X}$                                                                                                                 \\
            $ \mathcal{Y} $                            & Conjunto de clases del conjunto de datos $\mathcal{X}$, esto es, $\textit{K} = \lvert \mathcal{Y} \rvert$. \newline i.e. \scriptsize{${ \mathcal{Y} = \{1,2,...,\textit{K}\}}$} \\
            $ K $                                      & Número finito de clases de la etiqueta $y^{(i)} \in \mathcal{X}$ \newline i.e. \scriptsize{El número de clases o categorías de una clasificación}                               \\
            $ H $                                      & Conjunto de validación, este conjunto contiene muestras de todas las clases                                                                                                     \\
            $ \eta $                                   & Tasa de aprendizaje                                                                                                                                                             \\
            $ \hat{y} $                                & Predicción del modelo                                                                                                                                                           \\
            $ \theta $                                 & parámetros del modelo                                                                                                                                                           \\
            $ \mathcal{L} $                            & Función de perdida                                                                                                                                                              \\
            $ \phi ,\varphi $                          & Función de activación                                                                                                                                                           \\
            $ f\left(\cdot\right) $                    & Modelo                                                                                                                                                                          \\

            \hline
        \end{tabularx}
        \caption{Notación Parte 4}
        \label{tab:nnotation-part-4}
    \end{center}
\end{table}

\begin{table}[H]
    \begin{center}
        \begin{tabularx}{\textwidth}{|r|X|}
            \hline
            \textbf{Símbolo} & \textbf{Definición}                                 \\
            \hline % Notación de GANS (redes neuronales adversariales)
            $ x $            & Entrada original (limpia, sin modificar) de un dato \\
            $ x^{\prime} $   & Dato adversarial                                    \\
            $ y^{\prime} $   & Clase objetivo de ejemplo adversario                \\
            $ \delta $       & Perturbación generada por el algoritmo adversarial  \\
            % $ \Delta, \epsilon  $ & Restricción en la perturbación                                                   \\
            \hline
        \end{tabularx}
        \caption{Notación Parte 5}
        \label{tab:nnotation-part-5}
    \end{center}
\end{table}



% \newpage
% \KOMAoptions{paper=landscape,pagesize}
% \recalctypearea
% \section{This is my Landscape Page}
% Text in my landscape section\footnote{Footnote in Landscape}

% \newpage
% \KOMAoptions{paper=portrait,pagesize}
% \recalctypearea


\begin{enumerate}
    \item $\mathcal{Y}$ es el conjunto de clases del conjunto de datos $\mathcal{X}$.
    \item $\hat{c}(\mathcal{X})$ es el término que nos referimos al clasificador del conjunto $\mathcal{X}$
    \item $P(C|A)$ nos referimos a la probabilidad condicionada ($P$) de la clase ($C$) con los atributos ($A$).
          % \item $ a \gg b$ `a' mucho mayor que `b'.
          % \item $ a \ll b$ `a' mucho menor que `b'.
\end{enumerate}


\chapter*{Teoremas}
\addcontentsline{toc}{chapter}{\protect\numberline{}Teoremas}

% Maldición de la dimensionalidad -> clasificación basaa en naive bayes 
% Clasificador de Naïve Bayes
% Cada atributo como una variables independiente de la case
% Naïve significa ingenuo

\begin{theorem}[Bayes]
    \label{theorem:bayes}
    Sean $C$, $A$ dos eventos, y $P(C|A)$ la probabilidad de $C$ dependiente de $A$. Entonces.
    \[ P(C|A) = \frac{P(C|A) P(C)}{P(A)} \]
\end{theorem}

% http://www.lcc.uma.es/~jmortiz/archivos/Tema4.pdf
\begin{theorem}[Convergencia de perceptrón]
    \label{theorem:ConvergenciaRosenblatt}
    Si el conjunto de patrones de entrenamiento $ \{x^{1}, z^{1}\}, \{x^{2}, z^{2}\}, \cdots, \{x^{n}, z^{n}\} $ es linealmente separable entonces el Perceptrón simple encuentra una solución en un número finito de iteraciones, es decir, consigue que la salida de la red coincida con la salida deseada para cada uno de los patrones de entrenamiento.
\end{theorem}
% \clearpage\thispagestyle{empty}\cleardoublepage

\chapter{INTRODUCCIÓN}
\label{ch:1}

% Descripción de los objetivos del aprendizaje adversario:
\section{Introducción}

El objetivo principal de este trabajo es el análisis de los distintos tipos de ataques y defensas que se pueden aplicar a un proceso de aprendizaje automático, nos referimos principalmente al aprendizaje profundo.
Este trabajo tratará principalmente de clasificar y explorar la seguridad que cuentan los modelos de \gls{AI} actuales.
Existen una gran variedad de modelos y objetivos en redes neuronales, los ataques que se pueden hacer sobre estos modelos están orientados en las siguientes categorías (evadir, envenenar, explorar o denegar) el modelo.

Debemos comprender que debemos saber que la seguridad repercute en todo el proceso de creación del modelo, desde la recogida de los datos, tratamiento, diseño y creación del modelo.
Un ataque puede estar dirigido al conjunto de datos con el que se entrenará el modelo o a la red neuronal una vez entrenada.
Además, un ataque puede dirigirse a descubrir muestras que produzcan resultados erróneos, por lo que los posibles vectores de ataque son muy variados y complejos.

La idea de hacer robustos los modelos es que las defensas detecten ataques a la vez que se mejora la solidez del aprendizaje, para no cometer fallos al introducir valores anómalos.

Cada uno de estos tipos de ataques tiene su nomenclatura y debe definirse correctamente, ya que de lo contrario puede ser muy ambiguo el tipo de ataque que se está realizando, el objetivo que busca y los métodos que están empleando.

Lo que buscaremos en este trabajo será definir, analizar y estructurar los distintos tipos de ataques y sus posibles defensas, con el objetivo final de proteger y defender los modelos de inteligencia artificial para hacerlos más robustos.
Además de la construcción de una guía que pueda orientar a modelos más seguros y éticos.


\section{Motivación}

En la última década, se han logrado significativos avances en el campo de la inteligencia artificial. Sin embargo, a lo largo de este proceso, como suele ser habitual, se ha descuidado  aspectos cruciales relacionados con la seguridad. Esto ha dado lugar a la creación de productos que implementan la inteligencia artificial, pero presentan vulnerabilidades, riesgos potenciales, como redes neuronales poco robustas, filtraciones de datos, incumplimiento normativo, modelos que presentaban respuestas ofensivas, discriminantes ante etnias, etc. Además de presentar poca o ninguna explicabilidad de los resultados que presentan.

Esto lleva a muchos problemas de seguridad y riesgos que pueden afectar a productos que apliquen inteligencia artificial sin las medidas de adecuadas.

\begin{itemize}
    \item Alineación de la inteligencia artificial.
    \item Recopilación de datos.
    \item Bias y discriminación.
    \item Modelos poco robustos.
    \item Transparencia y explicabilidad.
    \item Escala de los modelos.
    \item Cumplimiento legal y normativo.
    \item Actualización continua.
    \item Detección de usos malintencionados.
\end{itemize}

Esto ha llevado a la creación de regulaciones de la inteligencia artificial que son muy vagas en sus conceptos de implementación.
Una primera aproximación fue el libro blanco\footnote{Se conoce como libros blancos a los documentos que publican los gobiernos en determinados casos para informar a los órganos legislativos o a la opinión pública con el objetivo de ayudar a los lectores a comprender un tema, resolver o afrontar un problema (por ejemplo diseñando una política gubernamental a largo plazo), o tomar una decisión. \href{https://es.wikipedia.org/wiki/Libro_blanco}{Enlace.}} de la inteligencia artificial en 2018 \cite{whitebook2020AI}

En este trabajo propondremos una guía similar a la matriz \gls{MITRE} con buenas prácticas para la construcción de modelos más robustos y que cumplan con nuestros objetivos.

% https://ec.europa.eu/commission/presscorner/detail/en/ip_23_6473
La necesidad de desarrollar un marco de inteligencia artificial fiable para todos los miembros de la unión europea llevo a la comisión europea a la creación de nuevas normativas con un enfoque basado en el riesgo.
Por lo que exploraremos el análisis de seguridad y posibles soluciones.


\section{Objetivos primarios}
% Estudiar la seguridad en la IA
% Experimentar con modelos inseguros
% Mejorar los modelos inseguros para una mejor robustez
El objetivo principal de este trabajo es el estudio actual de la seguridad en modelos de inteligencia artificial en redes neuronales profundas, búsqueda de vectores de ataque en modelos, posibles formas de auditar los distintos modelos y costes de ataques por distintos modelos adversariales que proponemos en la sección \ref{TODO}.


\section{Objetivos secundarios}
% Desarrollar un marco matriz como mitre con buenas prácticas
% Describir los estándares de seguridad actuales
Como objetivos secundarios se ha planteado el análisis y creación de un marco de trabajo para la implementación de modelos de inteligencia artificial fiable siguiendo los estándares normativos, legislativos y éticos que propone la unión europea, con el objetivo de facilitar la implementación de modelos fiables en los distintos estados miembros.

\section{Metodología y planificación del proyecto}

Una vez definidos los objetivos del tema de investigación y desarrollo, deberemos definir el alcance para estimar el tiempo requerido y el prosupuesto para el proyecto.
Recordemos que en la investigación puede ser del tipo exploratorio, buscando nuevos campos para alcanzar nuevos logros técnicos, confirmatoria para validar los resultados obtenidos en otras investigaciones o desarrollos, tambien puede ser una combinación de los tipos menecionados previamente.
En nuestro caso se tratará de un proyecto híbrido, tanto de exploración como de confirmación.

Haremos un repaso breve de las metodologias que recomienda la bibliografía existente, recordemos que las metodologias más comunes en las áreas de las ingenierias es una metodología cuantitativa.

\subsection{Planificación y costes}

Se ha de hacer una planificación estricta de las tareas, objetivos, hitos y dependencias, para ello usaremos los diagramas Gantt\footnote{Representación gráfica de la evolución de un proyecto. \href{https://asana.com/es/resources/gantt-chart-basics}{ASANA}} con el objetivo de representar los hitos de la investigación y del proyecto.
Seguiremos una metodología \textit{Scrum} y \textit{Lean} fijando reuniones cada semana para tener un control sobre el avance de la investigación y el desarrollo de este proyecto.

Primero definiremos en la tabla la lista de tareas, fecha de inicio, duración, fecha de fin y el coste de trabajo usando la técnica de tallas de camisetas.


\section{Estructura de la memoria}

% TODO

\subsection{Metodología}

% TODO

\subsection{Presupuesto}

% TODO
\clearpage\thispagestyle{empty}\cleardoublepage

\chapter{ANTECEDENTES Y ESTADO DEL ARTE}
\label{ch:2}
% ==============================================================================================================
\section{Introducción}

El proyecto que desarrollamos se encuentra en una frontera muy difusa de múltiples ramas del conocimiento, siendo muy interdisciplinar, se trata de una revisión de los ataques, seguridad y robustez a las redes neuronales.
Por lo que debemos explicar que es la ciencia de datos, el proceso \gls{KDD}, la inteligencia artificial generativa y la seguridad de la información.

\begin{figure}[H]
  \centering
  \centerline{\includesvg[width=0.75\columnwidth]{figures/ciencia-de-datos.drawio.svg}}
  \caption{Ciencia de datos como campo interdisciplinar.\\Fuente: Elaboración propia}
  \label{fig:ciencia-de-datos}
\end{figure}

Podemos dividir este trabajo en dos secciones muy relacionadas, la primera la inteligencia artificial y de segundo punto de importancia la seguridad de la información.
Desde sus inicios la inteligencia artificial, aunque con buenos resultados en muchos campos de aplicación resultaba en grandes fallos de seguridad, fiabilidad y robustez.
Por cómo están entrenadas las inteligencias artificiales (\acrshort{nn}) tiene múltiples puntos de ataque que son susceptibles de ser atacados, los principales son los datos, las arquitecturas o los pesos.
Ya que alterando cualquiera de estos componentes de forma se verá enormemente afectada el comportamiento.

% Ciencia de la computación
% Mineria
% Machine Learning
% Deep Learning
% GANs


\section{Antecedentes}

\subsection{Historia, línea temporal}

A continuación se describe una línea de temporal con los hitos más relevantes en el desarrollo de las redes neuronales adjuntando con referencias las investigaciones, articulos y publicaciones realizadas.
En este caso se trata de referencias en hitos de logros a nivel teórico como práctico.

\begin{vtimeline}[timeline color=cyan!80!blue, add bottom line, line offset=2pt, use timeline header,timeline title={Hitos de las redes neuronales artificiales}]
  1676        & The Chain Rule \cite{leibniz2012early}                                                    \endlr
  1847        & Augustin-Louis Cauchy \cite{lemarechal2012cauchy}                                         \endlr
  1943        & Threshold Logic Unit (TLU) \cite{mcculloch1943logical}                                    \endlr
  1949        & Teoría Hebbiana                                                                           \endlr
  1958        & Perceptron \cite{rosenblatt1958perceptron}                                                \endlr
  1959-1960   & Adaline y Madaline \cite{rosenblatt1958perceptron}                                        \endlr
  1965        & Multilayer Perceptron (MLP) \cite{baum1988capabilities}                                   \endlr
  1967-1968   & Deep Learning by Stochastic Gradient Descent \cite{karplus19671967}                       \endlr
  1980’s      & Neuronas Sigmoidales                                                                      \endlr
  ~           & Feedforward neural network (FNN) \cite{rumelhart1985learning}                             \endlr
  ~           & Backpropagation (BP) \cite{rosenblatt1962principles,etde_5080493,lecun1985learning}       \endlr
  1985        & Boltzmann Machine \cite{ACKLEY1985147}                                                    \endlr
  1987        & Adaptive resonance theory (ART) \cite{grossberg1987competitive}                           \endlr
  1989        & Convolutional neural networks (CNN) \cite{lecun1989backpropagation}                       \endlr
  ~           & Recurent neural networks (RNN) \cite{schmidhuber1993habilitation}                         \endlr
  1990        & Generative Adversarial Networks (GAN) as Game \cite{schmidhuberunsupervised}              \endlr
  1997        & Long short term memory (LSTM) \cite{Hochreiter1997LongSM, hochreiter1997long}                                 \endlr
  2006        & Deep Belief Networks (DBN) \cite{hinton2006fast}                                          \endlr
  ~           & Restricted Boltzmann Machine \cite{hinton2006reducing}                                    \endlr
  ~           & Encoder / Decoder (Auto-encoder) \cite{hinton2006reducing}                                \endlr
  2014        & Generative Adversarial Networks (GAN) Moderns \cite{6294131,goodfellow2014generative}     \endlr
  2018        & Style Generative Adversarial Networks (Style-GAN) \cite{karras2019stylebased}             \endlr
\end{vtimeline}


\newpage
\subsection{Historia de la Inteligencia Artificial}

Todo surge en 1676 por {Gottfried Wilhelm Leibniz} \ref{fig:gottfried-leibniz} publicó la regla de la cadena del cálculo diferencial, esencial para el análisis matemático, es la esencial para calcular como cambiará la función final si se cambian los pesos de funciones anteriores.

La regla de la cadena es fundamental para técnicas como el descenso de gradiente, propuesto por {Augustin-Louis Cauchy} en 1847 y utilizado para ajustar iterativamente los pesos de una NN durante el entrenamiento.
Posteriormente en 1805 {Adrien-Marie Legendre} y {Johann Carl Friedrich Gauss} desarrollaron \acrshort{nn}, matemáticamente eran regresiones lineales muy simples, similares a las redes neuronales lineales simples.
Esto lo uso {Gauss} para redescubrir el planeta enano Ceres.

\begin{figure}[H]
  \centering
  \includegraphics[width=0.45\textwidth]{figures/Gottfried_Wilhelm_Leibniz,_Bernhard_Christoph_Francke.jpg}
  \caption{Retrato de Gottfried Leibniz. \\Fuente: \href{https://es.wikipedia.org/wiki/Gottfried_Leibniz}{Wikipedia}}
  \label{fig:gottfried-leibniz}
\end{figure}

Aunque realmente la historia comienza en 1943 con la investigación de {Warren McCulloch} y {Walter Pitss}, publicaron el artículo \textit{A logical calculus of the ideas immanent in nervous activity} \cite{mcculloch1943logical}.
Dicho artículo creó distintas ramas de investigación (ordenadores digitales, inteligencia artificial, funcionamiento del perceptron).

\begin{figure}[H]
  \centering
  \includegraphics[width=0.65\textwidth]{figures/Warren Sturgis McCulloch Interview.png}
  \caption{Warren Sturgis McCulloch Interview. \\Fuente: \href{https://www.youtube.com/watch?v=8Wdz1Tj5084}{Entrevista en 1969}}
  \label{fig:Warren Sturgis McCulloch}
\end{figure}


En 1956 en la primera conferencia de inteligencia artificial organizada por la fundación {Rochester}, se reunen los investigadores fundadores de los conceptos actuales de la IA ({Minsky, McCarthy, Rochester, Shanon}), gran parte de la bibliografía se refiere a este punto como el origen y contacto de las redes neuronales artificiales.
En dicha conferencia (\textit{Nathaural Rochester}) presento el modelo de una red neuronal que fue el resultado de la investigación desarrollada por el equipo de investigación de IBM.

\begin{figure}[H]
  \centering
  \includegraphics[width=0.75\textwidth]{figures/conferecia 1956 - 1689170718524.png}
  \caption{Los padres de la inteligencia artificial. \\Fuente: \href{https://www.linkedin.com/pulse/first-ever-ai-conference-tracing-evolution-history-ofai-nicky-verd}{Linkedin}}
  \label{fig:conferencia-1956}
\end{figure}

En 1957 se presenta el ``\textit{Perceptron}'' por {Frank Rosenblatt}, dicho elemento es un sistema clasificador de patrones, además contaba con la capacidad de aprender, de ser robusto matemáticamente y poder adaptarse si algún componente se dañaba.

\begin{figure}[H]
  \centering
  \includegraphics[width=0.4\textwidth]{figures/perceptron.png}
  \caption{Mark I Perceptron. \\Fuente: \href{https://en.wikipedia.org/wiki/Perceptron}{Wikipedia}}
  \label{fig:perceptron}
\end{figure}

El \textit{Perceptron} fue diseñado originalmente para el reconocimiento óptico usando un sistema de 400 fotocélulas en rejilla.
Posteriormente se describió el problema de no-linealidad que presentaban los perceptrones (Problema XOR) \cite{cuevastello2018apuntes}.

De 1959 a 1960 {Bernard Widrow} y {Ted Hoff} desarollaron ``Adaline'' y ``Madaline'' \cite{widrow1960adaptive} que resolvía el problema de la no-linealidad y que tenía aplicación en el reconocimiento de voz, series temporales, caracteres, etc.

Posteriormente el \textit{MIT} realizo una investigación matemática muy crítica de todos los problemas que presentaba el \textit{Perceptron} llegando a la conclusión que tenían grandes problemas que no podrían ser resueltos, por lo que en la próxima década (años 60) se redujo drásticamente las investigaciones sobre el campo de las redes neuronales.
Esto llevo a uno de los famosos inviernos de la inteligencia artificial (1974 - 1980).

\begin{figure}[H]
  \centering
  \includegraphics[width=0.25\textwidth]{figures/Kunihiko Fukushima.jpg}
  \caption{Kunihiko Fukushima. \\Fuente: \href{https://www.ieice.org/eng/about_ieice/new_honorary_members_award_winners/2017/meiyo_05e.html}{IEICE}}
  \label{fig:kunihiko-fukushima}
\end{figure}

Durante la década de los 70 se hacen aportes a la teoría \textit{Hebbiana}, se aportan logros en al análisis y descripción de reglas adaptativas, además de otros aportes al principio de aprendizaje competitivo.
En 1979 presentó {Kunihiko Fukushima} la primera red neuronal \acrshort{cnn}, la llamó Neocognitron \cite{fukushima1979neural}, dicho trabajo en un futuro se mejoraría con técnicas de \gls{BP-NN}.

En la década de los 80 se realizaron aportes como el algoritmo de \gls{BP-NN} que surgio del artículo de {Hopfield} \cite{hopfield1982neural}, esto despertó la curiosidad de muchos investigadores a volver al campo de las redes neuronales.
Se realizaron aportes como las redes \acrshort{gnn}.
La investigación continuó con {Stephen Grossberg} que realizo aportes derivados de estudios fisiológicos de cómo funcionaban las neuronas y la plasticidad, lo que permitió la creación de reglas y postulados, esto se ve en los trabajos de las redes \acrshort{art-nn} \cite{grossberg1987competitive}.
La investigación de {Hopfield} basada en el trabajo de {Stephen Grossberg} creo un sistema computacional neuronal interconectado que tiende a un mínimo de energía.
En 1985 {David E. Rumelhart} basandose en la investigación realizada por {Paul Werbos} \cite{etde_5080493} realizo un análisis experimental del algoritmo \gls{BP-NN} y su aplicación en redes \acrshort{fnn} \cite{rumelhart1985learning}.

{Yann LeCun} junto a su equipo, en 1989 crearon la primera aplicación \acrshort{cnn} con técnicas \acrshort{bp} dicha aplicación podía reconocer números a partir de imágenes.

\begin{figure}[H]
  \centering
  \includegraphics[width=0.65\textwidth]{figures/yann-lecun - EyIwmEDW8AIQs1C.jpeg}
  \caption{Adaptive Systems Research Department at Bell Labs 1989. \\Fuente: \href{https://twitter.com/ylecun/status/1378718317695934465}{Twitter Yann Lecun}}
  \label{fig:adaptive-systems-research-department-at-bell-labs}
\end{figure}

En la década de los 90 se presentaron múltiples investigaciones y muchos avances en el campo, uno de los más importantes fue la presentación de la primera \acrshort{gan} como una curiosidad, ya que se presentó como un duelo entre dos redes neuronales, en un principio fue un generador probabilístico y un predictor con el objetivo de maximizar la pérdida de cada uno en un juego \textit{minimax}.

En 1991 se presentó el trabajo \textit{Predictability Minimization} \cite{urgen1991learning} dichas técnicas sirvieron de inspiración para el aprendizaje por refuerzo,
En marzo de 1991 se hizo una aproximación a los \textit{transformers} con auto atención, lograron separar el conocimiento del control como una máquina clásica, pero de una forma completamente neuronal, además de gestionar actualizaciones de los pesos de forma muy rápida y eficiente.

Durante la década de 1990 las redes neuronales tendían a ser muy sencillas, con pocas capas y no muy complejas por las limitaciones técnicas de la época.
Por lo que muchos investigadores propusieron soluciones similares a las redes \acrshort{rnn} que permitían una retroalimentación, además de aceptar secuencias de información arbitraria.
Otros propusieron soluciones como la jerarquía de \acrshort{rnn} autosupervisada que aprende representaciones en distintos niveles de abstracción.
Comienzan a proponerse redes similares a las que en un futuro se llamarían \acrshort{dbn} como un método no supervisado para \acrshort{fnn}.

En junio de 1991 {Sepp Hochreiter} \ref{fig:sepp-hochreiter} implemento el primer compresor de redes neuronales, además demostró uno de los principales problemas de las \acrshort{nn} el llamado problema del desvanecimiento o explosión del gradiente que hacía que el aprendizaje fallará.
Un análisis posterior condujo a los investigadores a una primera aproximación \acrshort{lstm}, aunque no sería hasta 1997 con la revisión por pares y publicación del artículo \textit{Long short-term memory} \cite{hochreiter1997long} que se solucionaría parcialmente el problema.

\begin{figure}[H]
  \centering
  \includegraphics[width=0.65\textwidth]{figures/Sepp Hochreiter.jpg}
  \caption{Sepp Hochreiter. \\Fuente: \href{https://people.idsia.ch/~juergen/fundamentaldeeplearningproblem.html}{IDSIA}}
  \label{fig:sepp-hochreiter}
\end{figure}

Más adelante en el 2014 {Goodfellow} \ref{fig:gan-ian-goodfellow} presento la primera red neuronal \acrshort{gan} pura para la generación de imágenes mediante el enfrentamiento de una red neuronal generativa contra una red neuronal discriminante entrenadas con el mismo conjunto de datos \cite{goodfellow2014generative}.
Durante los próximos años se realizaron muchos aportes a las redes neuronales generativas, principalmente de paralelización de los cálculos, técnicas de estabilización, generación condiciona, arquitecturas más eficientes, funciones de pérdidas más adecuadas, aplicaciones específicas (cambiar el estilo de pintura), redes apiladas, etc.
Fruto de todo ello {NVIDIA} en 2018 presento \gls{StyleGAN} \cite{karras2019stylebased} aunque publicaron el código en 2019 con fuertes mejoras.

\begin{figure}[H]
  \centering
  \includegraphics[width=0.2\textwidth]{figures/gan-goodfellow.png}
  \caption{Ian Goodfellow. \\Fuente: \href{https://www.technologyreview.es/s/10016/el-senor-de-las-gan-el-hombre-que-dio-imaginacion-las-maquinas}{MIT Technology Review}}
  \label{fig:gan-ian-goodfellow}
\end{figure}




\subsection{Privacidad}

\subsection{Explicabilidad}

\subsection{Normativa y estándares}
% https://eur-lex.europa.eu/resource.html?uri=cellar:e0649735-a372-11eb-9585-01aa75ed71a1.0008.02/DOC_1&format=PDF
% https://eur-lex.europa.eu/resource.html?uri=cellar:e0649735-a372-11eb-9585-01aa75ed71a1.0008.02/DOC_2&format=PDF

% Según la unión europea toda la seguridad es este apartado :ok: 
% (51) La ciberseguridad es fundamental para garantizar que los sistemas de IA resistan a las actuaciones de terceros maliciosos que, aprovechando las vulnerabilidades del sistema, traten de alterar su uso, conducta o funcionamiento o de poner en peligro sus propiedades de seguridad. Los ciberataques contra sistemas de IA pueden dirigirse contra elementos específicos de la IA, como los conjuntos de datos de entrenamiento (p. ej., contaminación de datos) o los modelos entrenados (p. ej., ataques adversarios), o aprovechar las vulnerabilidades de los elementos digitales del sistema de IA o la infraestructura de TIC subyacente. Por lo tanto, para asegurar un nivel de ciberseguridad adecuado a los riesgos, los proveedores de sistemas de IA de alto riesgo deben adoptar medidas adecuadas teniendo también en cuenta, cuando proceda, la infraestructura de TIC subyacente.









% https://www.youtube.com/@felipebravom



% Feb 1990: Generative Adversarial Networks / Curiosity Generative Adversarial Networks (GANs) have become very popular.[MOST] They were first published in 1990 in Munich under the moniker Artificial Curiosity. [AC90-20][GAN1] Two dueling NNs (a probabilistic generator and a predictor) are trying to maximize each other's loss in a minimax game.[AC](Sec. 1) The generator (called the controller) generates probabilistic outputs (using stochastic units[AC90] like in the much later StyleGANs[GAN2]). The predictor (called the world model) sees the outputs of the controller and predicts environmental reactions to them. Using gradient descent, the predictor NN minimizes its error, while the generator NN tries to make outputs that maximize this error: one net's loss is the other net's gain.[AC90] (The world model can also be used for continual online action planning.[AC90][PLAN2-3][PLAN])

% 4 years before a 2014 paper on GANs,[GAN1] my well-known 2010 survey[AC10] summarised the generative adversarial NNs of 1990 as follows: a "neural network as a predictive world modelis used to maximize the controller's intrinsic reward, which is proportional to the model's prediction errors" (which are minimized).

% The 2014 GANs are an instance of this where the trials are very short (like in bandit problems) and the environment simply returns 1 or 0 depending on whether the controller's (or generator's) output is in a given set.[AC20][AC][T22](Sec. XVII)

% Other early adversarial machine learning settings[S59][H90] were very different—they neither involved unsupervised NNs nor were about modeling data nor used gradient descent.

% The 1990 principle has been widely used for exploration in Reinforcement Learning[SIN5][OUD13] [PAT17][BUR18] and for synthesis of realistic images,[GAN1,2] although the latter domain was recently taken over by Rombach et al.'s Latent Diffusion, another method published in Munich,[DIF1] building on Jarzynski's earlier work in physics from the previous millennium[DIF2]  and more recent papers.[DIF3-5]

% In 1991, I published yet another ML method based on two adversarial NNs called Predictability Minimization for creating disentangled representations of partially redundant data, applied to images in 1996.

\section{Estado del arte}
% TODO

\subsection{La ciencia de datos}
% TODO

\subsection{La extracción del conocimiento}
% TODO
% Fases de la extraccón del conocimiento 
% KDD

\subsection{Aprendizaje automático - \textit{Machine Learning}}
% TODO

\subsection{Aprendizaje profundo - \textit{Deep Learning}}
% TODO

\subsection{Redes generativas adversariales - \textit{Generative Adversarial Networks (GAN)}}
% TODO

\subsection{La seguridad informática en redes neuronales}

El estado del arte de la seguridad es muy complejo por la constante evolución, las ciberamenazas están en constante evolución, los delincuentes informaticos adoptan nuevas técnicas y tácticas para eludir las medidas de seguridad tradicionales.
Limitaremos el alcance del estado del arte de la seguridad informática a los campos relacionados con la inteligencia artificial.

Amenazas de las redes neuronales.
% TODO

\begin{itemize}
  \item Envenenamiento:
  \item Inferencia:
  \item Extracción:
  \item Evasión:
\end{itemize}

\begin{figure}[H]
  \centering
  \centerline{\includesvg[width=1\columnwidth]{figures/adversarial-threats.drawio.svg}}
  \caption{Amenazas adversariales. \\Fuente: Elaboración propia}
  \label{fig:art-adversarial-threats}
\end{figure}


% TODO Red and Blue team

\begin{figure}[H]
  \centering
  \centerline{\includesvg[width=1\columnwidth]{figures/ART-for-red-and-blue-teams.drawio.svg}}
  \caption{Ataques y defensas en redes neuronales. \\Fuente: Elaboración propia}
  \label{fig:art-for-red-and-blue-teams}
\end{figure}


\subsubsection{Vectores de ataque en \textit{deep learning}}

La idea detrás del aprendizaje automático es la de poder predecir modelos predictivos, para ello se usan conjuntos de datos para el entrenamiento.

Los conjuntos de datos se han de procesar, ya que suelen contener mucho ruido, es decir, información muy poco valiosa, errónea o, por el contrario, una alta dimensionalidad de los datos que puede ser contraproducente por no poder reproducir esas medidas o por la poca información que aportan.

Los ataques al aprendizaje automático o profundo suelen estar dirigidos a las distintas etapas de creación y uso de un modelo.

\begin{itemize}
  \item Alteración de los datos de entrada.
  \item Alteración del proceso de aprendizaje.
  \item Extración de los datos de entrenamiento.
  \item Bloqueo del modelo.
\end{itemize}

Durante el transcurso del trabajo discutiremos los siguientes ataques y defensas que se pueden usar en los modelos.

\begin{itemize}
  \item Envenenamiento de datos.
  \item Puerta trasera.
  \item Extracción de datos
  \item Ingeniería inversa.
  \item Evasión.
\end{itemize}

Un buen símil con la seguridad clásica puede ser categorizar la seguridad de los modelos con el modelo \gls{STRIDE} de microsoft\footnote{Modelado de amenazas de microsoft \href{https://learn.microsoft.com/es-es/azure/security/develop/threat-modeling-tool-threats}{STRIDE}} o con la matriz \gls{MITRE}.

\begin{table}[H]
  \centering
  \small
  \def\arraystretch{1.5}
  \begin{tabular}{lp{10cm}}
    \toprule
    \textbf{Amenaza}    & \textbf{Tipo de ataque}                                           \\
    \midrule
    Corrupción de datos & Ataques de envenenamiento de datos, ataques de puerta trasera     \\
    Extracción de datos & Ataques de inferencia de membresía, ataques de ingeniería inversa \\
    Denegación          & Ataques de envenenamiento de datos                                \\
    \bottomrule
  \end{tabular}
  \caption{Amenazas STRIDE relacionadas con la seguridad de la IA}
  \label{tab.amenazas}
\end{table}

\subsubsection{Defensas contra los ataques adversarios}

\clearpage\thispagestyle{empty}\cleardoublepage

% \chapter{OBJETIVOS}
% http://eps-anterior.ujaen.es/TFMtemporal/tfmAsignados.php

% Hoy en día la Inteligencia Artificial y sus aplicaciones están cada vez más implantadas en diferentes sistemas de nuestra sociedad. Dentro de esta disciplina el destacan el campo del Machine Learning (Aprendizaje Automático), donde como resultado de aplicar algoritmos de aprendizaje a datos se obtienen modelos que destacan por los resultados que obtienen. Gracias a su precisión, estos modelos se encuentran desplegados en sistemas informáticos de gran importancia y que controlan procesos en diversos ámbitos de nuestra vida.
% Evidentemente, estos sistemas pueden son vulnerables ataques de seguridad siendo los denominados modelos adversarios (basados normalmente en redes neuronales) los que suelen estar implicados tanto en estos ataques como en el posible robustecimiento de los sistemas atacados y, por tanto, de los modelos de aprendizaje automático en los que se basan. El objetivo de este trabajo es hacer un estudio bibliográfico del campo de los modelos adversarios aplicados a la seguridad informática. Posteriormente, se elegirá un área aplicación, con sus respectivos conjuntos de datos y modelos representativos. Se realizarán experimentaciones en esta área y se analizarán los resultados.

% == Conocimientos Previos

% Los impartidos en el máster en Seguridad Informática además de básicos en aprendizaje automático.

% == Objetivos del TFM

% Estudio de los conceptos clave del campo del aprendizaje automático y del aprendizaje profundo en particular.
% Estudio de redes neuronales adversarias.
% Estudio de áreas de aplicación de estas redes en seguridad informática.
% Adaptación y ejecución de métodos de redes neuronales adversarias en un área de aplicación.
% Evaluar, comparar y analizar los resultados obtenidos

% == Metodología a Desarrollar

% Realizar una revisión bibliográfica del aprendizaje automático y del aprendizaje profundo
% Realizar una revisión bibliográfica del campo de los métodos de redes neuronales adversarias empleadas en seguridad informática.
% Estudiar las herramientas y bibliotecas existentes para el uso de redes neuronales adversarias.
% Estudiar y seleccionar un conjunto de datos para realizar la experimentación
% Diseñar la experimentación a realizar
% Adaptación y aplicación de métodos de redes adversarias al dataset seleccionado
% Analizar resultados y obtener conclusiones del trabajo realizado
% Escribir la memoria del trabajo realizado

En este capítulo explicaremos brevemente los objetivos que se quieren abordar en el proyecto, describiremos los objetivos principales como los específicos y su finalidad.



\section{Objetivos principales}

% El objetivo principal de este trabajo es hacer un estudio bibliográfico del campo de las redes neuronales adversarias aplicadas a la seguridad informática, búsqueda de vectores de ataque a modelos o a técnicas de seguridad clásicas, posibles formas de auditar y defender los distintos modelos. Además

% Nos centraremos en el estudio de los modelos generativos como \gls{GAN} o \gls{VAE}. Además de hacer un recorrido por la literatura actual de la \gls{IA} en la ciberseguridad.

El objetivo principal de este proyecto es el de realizar un estudio de las redes neuronales adversarias en el campo de la seguridad informática y su aplicación en la seguridad informática.

Realizar un estudio de la aplicación de redes neuronales en seguridad informática y analizar vectores de ataque a modelos de redes neuronales. Especialmente nos vamos a centrar en las redes neuronales adversarias y modelos generativos como son \gls{GAN} y \gls{VAE}, aunque existen otros modelos generativos, acotaremos la revisión bibliográfica a las \gls{GAN} únicamente y mencionaremos otros modelos con características similares.

Se adaptará y ejecutará métodos de redes neuronales adversarias en el área de seguridad, específicamente se ha planteado la temática de generar un modelo que sea capaz de evadir las medidas de seguridad clásicas de biometría. Es decir, usaremos una red neuronal adversaria para generar instancias falsas que sean capaces de evadir parcial o totalmente un sistema biométrico. El objetivo de esto es buscar si es posible evadir estas medidas de seguridad. Además, se deberá evaluar, comparar y analizar los resultados.

Se ha desarrollado un estudio de posibles vectores de ataque a algoritmos de biometría empleando estas arquitecturas y de como podemos optimizar la generación de estas instancias. Esto con el objetivo de generar instancias que puedan llegar a dar falsos positivos en sistemas de autenticación o identificación biométrica.

El objetivo de este proyecto es el de analizar si la seguridad informática puede ser comprometida desde distintos puntos, desde vulnerar un modelo que tenía fines legítimos o desde un atacante que utilice un modelo con fines maliciosos.

\section{Objetivos específicos}

% Como objetivos secundarios se ha planteado el análisis y creación de un marco de trabajo para la implementación de modelos de inteligencia artificial fiable siguiendo los estándares normativos, legislativos y éticos que propone la Unión Europea, con el objetivo de facilitar la implementación de modelos fiables en los distintos Estados miembros.


\begin{itemize}
    \item \textbf{Búsqueda de una visión general de la ciencia de datos actual}: Estudio del proceso KDD, aprendizaje profundo, redes neuronales, modelos y tareas. Nos enfocaremos en las redes neuronales adversarias aplicadas a la seguridad informática, centrándonos en redes generativas adversarias (\gls{GAN}) y brevemente en los (\gls{VAE}). Además, se analizará el estado normativo de estas aplicaciones dentro de la Unión Europea.
    
    % _TODO_: añadir este objetivo con este método.
    % OBJETIVO: Estudio de áreas de aplicación de estas redes en seguridad informática.
    % MÉTODO:   Realizar una revisión bibliográfica del campo de los métodos de redes neuronales adversarias empleadas en seguridad informática.
    
    \item \textbf{Análisis de vectores de ataque en redes neuronales.} Se analizarán los vectores de ataques específicos en la cadena de suministro que pueden ser comprometidos dentro de nuestro campo de estudio.
  
    \item \textbf{Implementar un modelo de red neuronal generativa adversaria que genere instancias sintéticas.} Con el fin de analizar la capacidad de las redes neuronales para evadir la seguridad, se creará un modelo con la tarea de generar instancias falsas que pueda provocar falsos positivos en sistemas de seguridad biométrica.
    
    \item \textbf{Optimizar la generación y eficiencia de instancias sintéticas.} Se buscará mejorar las técnicas de generación de instancias sintéticas, se explorará técnicas de optimización.
    
    \item \textbf{Evaluación del modelo mediante experimentos para extraer métricas.} Se deberá evaluar los resultados obtenidos por el modelo, analizando las métricas del modelo y analizando la evaluación de las instancias frente a algoritmos de extracción y coincidencia de características. 

    \item \textbf{Documentación y guía de instalación para la creación del modelo.} Se redactará la documentación de uso e instalación del modelo y las herramientas usadas durante el proyecto. Estos recursos estarán disponibles en el repositorio oficial del proyecto.
    % TODO Incluir cita del repositorio.
    
\end{itemize}
% \clearpage\thispagestyle{empty}\cleardoublepage

% \chapter{MATERIALES Y MÉTODOS}

\begin{lstlisting}[language=docker-compose-2,caption={Example docker-compose.yml},breaklines=true,label={code:compose}]
version: '2'
services:
  web:
     build: .
     ports:
        - "5000:5000"
     volumes:
        - .:/code
        - logvolume01:/var/log
     links:
        - redis
  redis:
     image: redis
volumes:
    logvolume01: {}
\end{lstlisting}


\section{Análisis}

\section{Ataques}

\section{Defensas}


% \clearpage\thispagestyle{empty}\cleardoublepage

% \input{5_Resultados.tex}
% \clearpage\thispagestyle{empty}\cleardoublepage

% \chapter{CONCLUSIONES}

% \cite{devops-redhat}

% \emoji{woman-health-worker-medium-skin-tone}


% Ââ Êê Îî Ôô Ŵŵ Ŷŷ Ïï

% Ăă Ĕĕ Ĭĭ Ŏŏ Ŭŭ

% Āā Ēē Īī Ōō Ūū Ȳȳ

% \gls{latex}
% \clearpage\thispagestyle{empty}\cleardoublepage

\clearpage

\printglossaries
% ===== Configuración de la bibliografía

\pagenumbering{roman}  % Numeración romana para el índice
\bibliographystyle{unsrtnat}
\bibliography{bibliography.bib}
\addcontentsline{toc}{chapter}{Bibliografía}
\nocite{*}

\end{document}