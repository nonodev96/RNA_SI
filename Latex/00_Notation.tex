\chapter{Notación}

\begin{table}[ht!]
    \begin{center}
        \begin{tabularx}{\textwidth}{|l|X|}\hline
            \textbf{Simbolo}                                                                                                     & \textbf{Definición}                                                                                                                                                         \\ \hline
            $\mathbb{Z}$                                                                                                         & Números enteros                                                                                                                                                             \\
            $\mathbb{Z^{+}}$                                                                                                     & Números enteros positivos                                                                                                                                                   \\
            $\mathbb{N}$                                                                                                         & Números naturales                                                                                                                                                           \\
            $\mathbb{R}$                                                                                                         & Números reales                                                                                                                                                              \\
            $\mathbb{C}$                                                                                                         & Números complejos                                                                                                                                                           \\
            $R^{n}$                                                                                                              & Espacio vectorial $n$-dimensional de números reales                                                                                                                         \\
            $\mathcal{X}$                                                                                                        & Conjunto de datos de entrada que es usado para entrenar la red neuronal (\gls{DNN}), donde $\mathcal{X} \in \mathbb{R}^{N}$                                                 \\

            % $\mathcal{A} \backslash \mathcal{B}  $ & $\mathcal{A}$ sin $\mathcal{B}$, el conjunto de elementos de $\mathcal{A}$, pero no en $\mathcal{B} $                       \\
            % $\exists{x}$                           & Existe $x$                                                                                                                  \\
            % $\forall{x}$                           & Para todo $x$                                                                                                               \\
            % $ g \circ f $                          & Función composición (g después de f)                                                                                        \\
            % $ a := b $                             & $a$ se define como $b$                                                                                                      \\
            % $ a \propto  b $                       & $a$ proporcional a $b$                                                                                                      \\
            $I_{m}$                                                                                                              & Matriz identidad de tamaño $ m \times m$                                                                                                                                    \\
            $0_{m,n}$                                                                                                            & Matriz de ceros de tamaño $ m \times n$                                                                                                                                     \\
            $\underline{0}_{n}$                                                                                                  & Vector de ceros de tamaño $n$                                                                                                                                               \\
            $1_{m,n}$                                                                                                            & Matriz de unos de tamaño $ m \times n$                                                                                                                                      \\
            $\underline{1}_{n}$                                                                                                  & Vector de unos de tamaño $n$                                                                                                                                                \\
            $e_{i}$                                                                                                              & Vector estandar o vector canonico. i.e. ${\scriptscriptstyle v_{x} = (1,0,0), v_{y} = (0,1,0), v_{z} = (0,0,1)}$                                                            \\
            $rk(\textbf{\textit{A}})$                                                                                            & Rango de la matriz $\textit{\textbf{A}}$                                                                                                                                    \\
            $Im(\Phi)$                                                                                                           & Imagen del mapeo lineal $\Phi$                                                                                                                                              \\
            $ker(\Phi)$                                                                                                          & Núcleo (espacio nulo) de un mapeo lineal $\Phi$                                                                                                                             \\
            $span[\textbf{b}_{1}]$                                                                                               & Sistema generador de $\textbf{b}_{1}$                                                                                                                                       \\
            $tr(\textbf{\textit{A}})$                                                                                            & Traza de $\textit{\textbf{A}}$                                                                                                                                              \\
            $det(\textbf{\textit{A}})$                                                                                           & Determinante de $\textit{\textbf{A}}$                                                                                                                                       \\
            $\lvert·\rvert$                                                                                                      & Determinante o valor absoluto                                                                                                                                               \\
            $\lvert\lvert·\rvert\rvert$                                                                                          & Norma euclidiana                                                                                                                                                            \\
            $\underline{x} \odot \underline{m}$                                                                                  & Operación por elementos de los vectores o matrices                                                                                                                          \\
            $\langle \underline{z}, \underline{y}\rangle  = \underline{z}' \underline{y} = \textstyle{\sum_{j=1}^{N}{z_j y_j}} $ & Producto escalar de vectores por columnas $\underline{z}, \underline{y} \in \mathbb{R}^{N}$                                                                                 \\
            ${\Vert \underline{x} - \underline{y} \Vert}_{q}$                                                                    & Distancia en la misma dimensión entre $\underline{x}$ e $\underline{y}$                                                                                                     \\\hline

            ${x_{1}, x_{2},...,x_{n}}$                                                                                           & Es el conjunto con $n$ elementos                                                                                                                                            \\\hline
            $x_{i}: a \in A$                                                                                                     & TODO                                                                                                                                                                        \\

            $\textbf{v} = [v_{i}]_{i=1,...,n}$                                                                                   & Vectores definidos en negrita y minuscula                                                                                                                                   \\
            $\textbf{V} = [v_{i,j}]_{i=1,...,n, j=1,...,m}$                                                                      & Matrices definidas en negrita y mayúscula                                                                                                                                   \\
            $N$                                                                                                                  & Dimensión del espacio muestral de entrada                                                                                                                                   \\
            $T = \lvert \mathcal{X} \rvert < \infty$                                                                             & Número finito de instancias del conjunto de datos $\mathcal{X}$                                                                                                             \\
            $K$                                                                                                                  & Número finito de clases de conjunto de datos $\mathcal{X}$ en problemas de clasificación                                                                                    \\
            $H$                                                                                                                  & Conjunto de validación, este conjunto contiene muestras de todas las clases                                                                                                 \\
            $\mathcal{Y}$                                                                                                        & Conjunto de clases del conjunto de datos $\mathcal{X}$, esto es, $\textit{K} = \lvert \mathcal{Y} \rvert$. i.e. ${\scriptscriptstyle \mathcal{Y} = \{1,2,...,\textit{K}\}}$ \\
            $f_{*} = \text{min}_x~f(x)$                                                                                          & El valor de función más pequeño de $f$                                                                                                                                      \\
            ${x}_x \in \text{arg min}_{x}~f(x) $                                                                                 & El valor $x_{*}$ que minimiza $f$ (conjunto de valores)                                                                                                                     \\
            $\mathcal{L}$                                                                                                        &                                                                                                                                                                             \\
                                                                                                                                 &                                                                                                                                                                             \\
                                                                                                                                 &                                                                                                                                                                             \\
                                                                                                                                 &                                                                                                                                                                             \\
                                                                                                                                 &                                                                                                                                                                             \\
                                                                                                                                 &                                                                                                                                                                             \\
                                                                                                                                 &                                                                                                                                                                             \\
                                                                                                                                 &                                                                                                                                                                             \\ \hline
        \end{tabularx}
        \caption{Notación}
        \label{tab:nnotation}
    \end{center}
\end{table}



\begin{enumerate}


    \item $y$ es el conjunto de clases del conjunto $\mathcal{X}$.
    \item $\hat{c}(\mathcal{X})$ es el término que nos referimos al clasificador del conjunto $\mathcal{X}$
    \item $P(C|A)$ nos referimos a la probabilidad condicionada ($P$) de la clase ($C$) con los atributos ($A$).
          % \item $ a \gg b$ `a' mucho mayor que `b'.
          % \item $ a \ll b$ `a' mucho menor que `b'.
\end{enumerate}


\chapter*{Teoremas}

% Maldición de la dimensionalidad -> clasificación basaa en naive bayes 
% Clasificador de Naïve Bayes
% Cada atributo como una variables independiente de la case
% Naïve significa ingenuo

\begin{theorem}[Bayes]
    \label{theorem:bayes}
    Sean $C$ y $A$ dos eventos, y $P(C|A)$ la probabilidad de $C$ dependiente de $A$. Entonces.
    \[ P(C|A) = \frac{P(C|A) P(C)}{P(A)} \]
\end{theorem}