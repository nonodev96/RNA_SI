% !TeX root = main.tex
\documentclass{beamer}
%Information to be included in the title page:
\usepackage[T1, T2A]{fontenc}% T2A for Cyrillic font encoding
\usepackage[spanish, english]{babel}
\usepackage{tikz}
\usepackage{multimedia}




\usepackage{actuarialangle}
\usepackage{csvsimple}

\usepackage{amsmath}
\usepackage{amssymb}
\usepackage{amsthm}
\usepackage{amsfonts}

\usepackage{commath}
\usepackage{bookmark} 
% Para figuras y espacios centrados con tamaño de linea pequeño y alineada las lineas
\usepackage[justification=centering,font=footnotesize]{caption}
\usepackage{subcaption}
\usepackage{varwidth}

% VTimeline
\usepackage{charter} 
\usepackage{environ}
\usepackage{tikz}
\usepackage{pgfplots}



\usepackage{tikz} 



\usetheme{Madrid}
% \usecolortheme{beaver}
\usecolortheme{default}
 


\title[SAI] %optional
{Redes neuronales adversarias en seguridad informática}

\subtitle{SAI}

\author[Antonio, Mudarra Machuca] % (optional, for multiple authors)
{M.~M.~Antonio\inst{1}}

\institute[UJAEN] % (optional)
{
  \inst{1}%
  Sistemas Inteligentes y Minería de Datos\\
  Escuela Politécnica Superior de Jaén\\
  Universidad de Jaén
}

\date[SAI 2024] % (optional)
{Seminario Seguridad en redes neuronales, April 2024}

\logo{\includegraphics[height=1cm]{figures/UJA.png}}


% https://es.overleaf.com/learn/latex/Beamer_Presentations%3A_A_Tutorial_for_Beginners_(Part_1)%E2%80%94Getting_Started
% https://es.overleaf.com/learn/latex/Beamer_Presentations%3A_A_Tutorial_for_Beginners_(Part_2)%E2%80%94Lists%2C_Columns%2C_Pictures%2C_Descriptions_and_Tables
% https://es.overleaf.com/learn/latex/Beamer_Presentations%3A_A_Tutorial_for_Beginners_(Part_3)%E2%80%94Blocks%2C_Code%2C_Hyperlinks_and_Buttons
% https://es.overleaf.com/learn/latex/Beamer_Presentations%3A_A_Tutorial_for_Beginners_(Part_4)%E2%80%94Overlay_Specifications
% https://es.overleaf.com/learn/latex/Beamer_Presentations%3A_A_Tutorial_for_Beginners_(Part_5)%E2%80%94Themes_and_Handouts

\begin{document}

\frame{\titlepage}

\begin{frame}
    \frametitle{Sample frame title}
    This is some text in the first frame. \\
    This is some text in the first frame. \\
    This is some text in the first frame.
\end{frame}

\begin{frame}
    \frametitle{There Is No Largest Prime Number}
    \framesubtitle{The proof uses \textit{reductio ad absurdum}.}
    \begin{theorem}
        There is no largest prime number.
    \end{theorem}
    \begin{proof}
        \begin{enumerate}
            \item<1-| alert@1> Suppose $p$ were the largest prime number.
            \item<2-> Let $q$ be the product of the first $p$ numbers.
            \item<3-> Then $q+1$ is not divisible by any of them.
            \item<1-> Thus $q+1$ is also prime and greater than $p$.\qedhere
        \end{enumerate}
    \end{proof}
\end{frame}


\begin{frame}
    \frametitle{There Is No Largest Prime Number}
    \framesubtitle{The proof uses \textit{reductio ad absurdum}.}

    \begin{block}{Mi texto}
        \begin{enumerate}
            \item<1-| alert@1> Suppose $p$ were the largest prime number.
            \item<2-> Let $q$ be the product of the first $p$ numbers.
            \item<3-> Then $q+1$ is not divisible by any of them.
            \item<4-> Thus $q+1$ is also prime and greater than $p$.\qedhere
        \end{enumerate}
    \end{block}

    \pause[4]
    \begin{example}
        Texto de ejemplo
    \end{example}
\end{frame}

\begin{frame}
    \centering
    \begin{math}
        \mathcal{L}_{D}^{GAN} =
    \end{math}
    % \movie[borderwidth=5pt,height=0.6\textwidth,width=0.8\textwidth,poster,showcontrols,autostart]{}{figures/output.avi}
\end{frame}

\end{document}