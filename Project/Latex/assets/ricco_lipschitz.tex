\documentclass{article}

\newtheorem{thm}{Theorem}
\newtheorem{dfn}{Definition}
\newtheorem{ex}{Example}

\title{Lipschitz Functions}
\author{Lorianne Ricco}

\begin{document}

\maketitle

\begin{dfn}
Let $f(x)$ be defined on an interval $I$ and suppose we can find two positive constants $M$ and $\alpha$  such  that

\begin{displaymath}
| f(x_{1}) - f(x_{2}) | \leq M |x_{1} - x_{2}|^\alpha 
\hbox{ for all } x_{1}, x_{2} \in I.
\end{displaymath}

Then $f$ is said to satisfy a \textit{Lipschitz Condition of order $\alpha$} and
we say that $f \in Lip(\alpha)$. 
\end{dfn}




\begin{ex}
Take $f(x) = x$ on the interval $[a, b]$. Then 
$$
|f(x_1) - f(x_2)| = |x_1 - x_2|
$$
That implies that
$f \in Lip(1)$.

Now take $f(x) = x^2$ on the interval $[a, b]$. Then 

$$
|f(x_1) - f(x_2)| = |{x_1}^2 - {x_2}^2| 
   = |x_1 - x_2| |x_1 + x_2| 
   \le M |x_1 - x_2|
$$

with $M = 2 max(|a|, |b|)$. Hence, again $f \in Lip(1)$.

The function $f(x) = 1/x $ on $(0, 1)$. Is it $Lip(1)$ ? How about $Lip(1/2)$?
How about $Lip(\alpha)$?
\end{ex}




\begin{thm}
$Lip (\alpha)$ is a linear space. 
\end{thm} 

\textbf{Proof}
We will look at a part of this proof.
Let $f,g \in Lip(\alpha)$.  

\begin{displaymath}
(f + g) (x) \in Lip(\alpha)
\end{displaymath}

Then,

\begin{displaymath}
|f(x) + g(x) - f(y) + g(y)| \leq M |x - y|^\alpha
\end{displaymath}

If $f \in Lip (\alpha)$ it implies that 

\begin{displaymath}
|f(x) - f(y)| \leq M_1 |x -y|^\alpha
\end{displaymath}

If $g \in Lip (\alpha)$ it implies that

\begin{displaymath}
|g(x) - g(y)| \leq M_2 |x - y|^\alpha
\end{displaymath}

If $(f + g) \in Lip (\alpha)$ it imples that

\begin{displaymath}
|(f + g)(x) - (f + g)(y)| \leq M_3 |x - y|^\alpha
\end{displaymath}

\begin{displaymath}
|f(x) + g(x) - f(y) - g(y)| =
\end{displaymath}

\begin{displaymath}
|f(x) - f(y) + g(x) - g(y)| =
\end{displaymath}

\begin{displaymath}
|f(x)  - f(y)| + |g(x) - g(y)| =
\end{displaymath}

By the triangle inequality,

\begin{displaymath}
|f(x) - f(y)| + |g(x) - g(y)| \leq
\end{displaymath}

\begin{displaymath}
M_1 |x - y|^\alpha + M_2 |x - y|^\alpha = |M_1 + M_2| |x - y|^\alpha
\end{displaymath}




\begin{thm}
If $f \in Lip (\alpha)$ with $\alpha > 1$ then 
$f = constant$.  
\end{thm}

\textbf{Proof}
Left as homework for everyone.

\section{Lipschitz and Continuity}

\begin{thm}
If $f \in Lip (\alpha)$ on I, then $f$ is continous; 
indeed, uniformly contiuous on $I$. 
\end{thm}

Last time we did continuity with $\epsilon$ and $\delta$. An alternative
definition of continuity familar from calculus is: $f$ is continuous at
$x = c$ if:

\begin{itemize}
\item $f(c)$ exists
\item $lim_{x \rightarrow c} f(x)$ exists
\item $lim_{x \rightarrow c} f(x) = f(c)$
\end{itemize}

In order to be continuous, if $|x - x_{0}| < \delta$, then $|f(x) - f(x_{0})| < \epsilon$.

\textbf{Proof}

\begin{displaymath}
|f(x) - f(c)| \leq M |x - c|^\alpha
\end{displaymath}

\begin{displaymath}
lim_{x \rightarrow c} |f(x) - f(c)| \leq M lim_{x \rightarrow c} |x -c|^\alpha = 0
\end{displaymath}

This implies 

\begin{displaymath}
lim_{x \rightarrow c} f(x) = f(c)
\end{displaymath}

How about continuous implies $Lip(\alpha)$?


\section{Lipschitz and Differentiability}

\begin{thm}
If $f \in Lip (\alpha)$, it may fail to be differentiable, but if it 
possesses a derivative satisfying $|f'(x)| \le M$ then $f \in Lip(1)$.  
\end{thm}

In order to be differentiable, 

\begin{displaymath}
\lim_{x \rightarrow \infty} {f(x) - f(x_0) \over x - x_0} = f'(x_0)
\end{displaymath}

By the Mean Value Theorem,

\begin{displaymath}
\Rightarrow {\left|f(x) - f(y)\right| \over \left|x-y\right|} = |f'(c)|, c \in (x,y)
\end{displaymath}

This implies that $$|f(x) - f(y)| = |f'(c)| |x-y|$$

Now if $$|f'(x)|$$ exsits and is bounded by $M$, then $$|f(x) - f(y)| \le M |x-y|$$ which implies $f \in Lip(1)$.  









\end{document}
