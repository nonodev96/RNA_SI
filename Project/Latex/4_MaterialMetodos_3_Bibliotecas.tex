\clearpage
\section{Bibliotecas para ciencia de datos}
% https://verneacademy.com/blog/articulos-ia/10-librerias-python-data-science-machine-learning/

Existe una gran variedad de bibliotecas y librerías que ofrecen soluciones a los principales problemas en el campo de la ciencia de datos y de la inteligencia artificial. A continuación mostramos las herramientas más usadas por la comunidad científica y cuáles son sus características.

\begin{figure}[H]
    \centering
    % Primera fila de imágenes
    \captionsetup{justification=centering}
    % \captionsetup[subfigure]{justification=justified}

    \begin{subfigure}{0.15\textwidth}
        \centering
        \includegraphics[width=1\textwidth]{figures/chapter04/lib/Pandas.png}
        \caption{Pandas}
        \label{fig:imagen1}
    \end{subfigure}
    \hfill
    \begin{subfigure}{0.15\textwidth}
        \centering
        \includegraphics[width=1\textwidth]{figures/chapter04/lib/NumPy.png}
        \caption{NumPy}
        \label{fig:imagen2}
    \end{subfigure}
    \hfill
    \begin{subfigure}{0.15\textwidth}
        \centering
        \includegraphics[width=1\textwidth]{figures/chapter04/lib/Plotly.png}
        \caption{Plotly}
        \label{fig:imagen3}
    \end{subfigure}
    
    \bigskip 
    
    % Segunda fila de imágenes
    \begin{subfigure}{0.15\textwidth}
        \centering
        \includegraphics[width=1\textwidth]{figures/chapter04/lib/Matplotlib.png}
        \caption{Matplotlib}
        \label{fig:imagen4}
    \end{subfigure}
    \hfill
    \begin{subfigure}{0.15\textwidth}
        \centering
        \includegraphics[width=1\textwidth]{figures/chapter04/lib/scikit-learn.png}
        \caption{scikit-learn}
        \label{fig:imagen5}
    \end{subfigure}
    \hfill
    \begin{subfigure}{0.15\textwidth}
        \centering
        \includegraphics[width=1\textwidth]{figures/chapter04/lib/shap.png}
        \caption{shap}
        \label{fig:imagen6}
    \end{subfigure}
    
    \caption{Bibliotecas de Python para ciencia de datos}
    \label{fig:cuadricula}
\end{figure}

% \begin{figure}[H]
%     \centering
%     \begin{subfigure}{0.15\textwidth}
%         \centering
%         \includegraphics[width=\textwidth]{figures/chapter04/lib/Pandas.png}
%         \caption{Pandas}
%         \label{fig:imagen1}
%     \end{subfigure}
%     \hspace{0.02\textwidth} % Espacio horizontal
%     \begin{subfigure}{0.15\textwidth}
%        \centering
%         \includegraphics[width=\textwidth]{figures/chapter04/lib/NumPy.png}
%         \caption{NumPy}
%         \label{fig:imagen2}
%     \end{subfigure}
    
%     \hspace{0.10\textwidth} % Espacio horizontal
    
%     \begin{subfigure}{0.15\textwidth}
%         \centering
%         \includegraphics[width=\textwidth]{figures/chapter04/lib/Plotly.png}
%         \caption{Plotly}
%         \label{fig:imagen3}
%     \end{subfigure}
%     \hspace{0.02\textwidth} % Espacio horizontal
%     \begin{subfigure}{0.15\textwidth}
%         \centering
%         \includegraphics[width=\textwidth]{figures/chapter04/lib/Matplotlib.png}
%         \caption{Matplotlib}
%         \label{fig:imagen4}
%     \end{subfigure}

%     % \vspace{0.02\textwidth} % Espacio vertical entre filas

%     \begin{subfigure}{0.15\textwidth}
%         \centering
%         \includegraphics[width=\textwidth]{figures/chapter04/lib/scikit-learn.png}
%         \caption{scikit-learn}
%         \label{fig:imagen5}
%     \end{subfigure}
%     \hspace{0.02\textwidth} % Espacio horizontal
%     \begin{subfigure}{0.15\textwidth}
%         \centering
%         \includegraphics[width=\textwidth]{figures/chapter04/lib/shap.png}
%         \caption{shap}
%         \label{fig:imagen6}
%     \end{subfigure}
%     % \hspace{0.02\textwidth} % Espacio horizontal
%     % \begin{subfigure}{0.15\textwidth}
%     %     \centering
%     %     \includegraphics[width=\textwidth]{figures/chapter04/lib/Azure.png}
%     %     \caption{AzureML}
%     %     \label{fig:imagen7}
%     % \end{subfigure}
%     % \hspace{0.02\textwidth} % Espacio horizontal
%     % \begin{subfigure}{0.20\textwidth}
%     %     \centering
%     %     \includegraphics[width=\textwidth]{figures/chapter04/lib/Azure.png}
%     %     \caption{Imagen 8}
%     %     \label{fig:imagen8}
%     % \end{subfigure}
    
%     \caption{Bibliotecas de python para ciencia de datos}
%     \label{fig:cuadricula}
% \end{figure}

% ====================================================================================================================
\subsection{Pandas}

\textbf{Pandas} es una de las bibliotecas más importantes en \textit{Data Science}, facilita el manejo de grandes volúmenes de datos, simplifica la lectura y escritura de múltiples fuentes. Esto lo realiza creando una estructura de datos tipo tabla con columnas llamadas \texttt{Series} de datos y operando sobre matrices.

Alguna de las características más relevantes de la librería son:

\begin{itemize}
    \item \textbf{Gestión de datos}: permite el tratamiento adecuado de valores faltantes.
    \item \textbf{Agrupación y ordenación}: facilita el agrupamiento y la ordenación de datos.
    \item \textbf{Manejo de series temporales}: incluye herramientas para manipular y analizar datos con índices de tiempo.
    \item \textbf{Mezcla y fusión de datos}: permite combinar conjuntos de datos de diversas fuentes.
    \item \textbf{Edición de estructura}: es posible agregar, eliminar o modificar filas y columnas de forma eficiente.
    \item \textbf{Interfaz de entrada y salida}: proporciona funciones para leer y escribir datos en múltiples formatos.
\end{itemize}

% ====================================================================================================================
\subsection{Numpy}

\textbf{NumPy} (Numerical Python) es una de las bibliotecas principales para el cálculo científico, gestiona grandes volúmenes de datos, operar con vectores y matrices multidimensionales, cuenta con cientos de funciones matemáticas de alto rendimiento, esta librería es la base en la que se apoyan otras bibliotecas.

Alguna de las características más relevantes de \texttt{numpy} son:

\begin{itemize}
    \item \textbf{Datos multidimensionales}: proporciona la estructura \texttt{ndarray}, que permite almacenar y manipular datos en forma de vectores o matrices de múltiples dimensiones de forma muy eficiente.
    \item \textbf{Alto rendimiento}: está implementado en \texttt{C} con una alta eficiencia.
    \item \textbf{Compatibilidad}: es compatible con pandas, SciPy, Matplotlib, etc.
    \item \textbf{Manipulación de datos}: permite reestructurar, aplanar y cambiar la forma, facilita el procesamiento de datos.
    \item \textbf{Operaciones matemáticas avanzadas}: cuenta con una amplia variedad de funciones matemáticas.
\end{itemize}

% ====================================================================================================================
\subsection{Plotly}

\textbf{Plotly} es una de las bibliotecas por referencia de generación de gráficos para la visualización de datos, genera gráficos dinámicos, interactivos, compatible con una gran variedad de bibliotecas, permite exportar los gráficos a distintos tipos de formatos, incluso se integra en aplicaciones web.

Cuenta con gráficos avanzados como contornos, mapas de calor, soporta gráficos de series temporales, gráficos de velas y cascada. Cuenta con visualización geo espacial. Plotly es útil para inteligencia artificial con gráficos de regresión y representación de curvas ROC, en bioinformática, con gráficos de volcanes y Manhattan, etc.

% ====================================================================================================================
\subsection{Mathplotlib}

\textbf{Mathplotlib} es una biblioteca para la generación de gráficas, se especializa en ciencia de datos, estadística e inteligencia artificial, en este caso solo soporta python y al igual que plotly cuenta con una gran variedad de gráficos, es compatible con otras herramientas, permite extender sus funciones como con otros frameworks como \textit{Seaborn}. Además, permite exportar en múltiples formatos, una de las pocas desventajas es que no cuenta con soporte para gráficas interactivas en web.

Aunque es mucho más completo y cuenta con muchas más características que plotly, puede ser más complejo, pues la personalización requiere un aprendizaje profundo de la biblioteca.

% ====================================================================================================================
\subsection{Scikit-learn}

\textbf{Scikit-learn} es la biblioteca para python especializada en inteligencia artificial, machine learning y minería de datos. Cuenta con una gran variedad de funciones, algoritmos, métricas e implementaciones de arquitecturas de modelos de inteligencia artificial.

Es una de las bibliotecas principales para la investigación  y cuenta con un gran apoyo por parte de la comunidad y una buena documentación.

Existe una derivación de la biblioteca que se especializa en el análisis y procesamiento de imágenes. \textbf{scikit-image} cuenta con muchas tareas de manipulación de color, canales, transformaciones geométricas, filtrado, restauración, segmentación de objetos, etc.

% ====================================================================================================================
\subsection{Adversarial Robustness Toolbox (ART)}

\textbf{Adversarial Robustness Toolbox} es una de las librerías de python para la seguridad del \textit{machine learning}, esta librería es muy completa y cuenta con el apoyo de cientos de investigadores a nivel mundial, pensada para el análisis de modelos de inteligencia artificial, implementa los algoritmos, métodos y técnicas con los que los investigadores logran evadir o atacar los modelos.

Su complejidad es alta, pues se requiere de un estudio profundo de como funciona los modelos, además de comprender el funcionamiento de la técnica que se quiere realizar.

Esta librería está apoyada por el grupo de investigadores de IBM, que mantiene y desarrolla el proyecto, investigadores pueden aportar sus descubrimientos implementándolos en la librería.

% ====================================================================================================================
\subsection{Shap}

\textbf{SHAP} es una de las bibliotecas para python especializada en la explicabilidad de modelos de inteligencia artificial, se usa para entender los resultados de los modelos de redes neuronales o de los modelos random forest, etc. La biblioteca permite la visualización de características, análisis globales, análisis de las predicciones, detectar sesgos, etc.


% ====================================================================================================================
\subsection{Otras librerías}

Existen una gran variedad de herramientas que puede ser útiles para la investigación científica, desarrollo e implementación de modelos.

\begin{multicols}{3} 
    \begin{itemize}
        % \item OpenCV
        \item AzureML
        \item Scipy
        \item Seaborn
        \item Imbalance Learning
        \item statsmodels
        \item NLTK
        \item LightGBM
        \item XGBoost
        \item AIX360
    \end{itemize}
\end{multicols}