\chapter{OBJETIVOS}
% http://eps-anterior.ujaen.es/TFMtemporal/tfmAsignados.php

% Hoy en día la Inteligencia Artificial y sus aplicaciones están cada vez más implantadas en diferentes sistemas de nuestra sociedad. Dentro de esta disciplina el destacan el campo del Machine Learning (Aprendizaje Automático), donde como resultado de aplicar algoritmos de aprendizaje a datos se obtienen modelos que destacan por los resultados que obtienen. Gracias a su precisión, estos modelos se encuentran desplegados en sistemas informáticos de gran importancia y que controlan procesos en diversos ámbitos de nuestra vida.
% Evidentemente, estos sistemas pueden son vulnerables ataques de seguridad siendo los denominados modelos adversarios (basados normalmente en redes neuronales) los que suelen estar implicados tanto en estos ataques como en el posible robustecimiento de los sistemas atacados y, por tanto, de los modelos de aprendizaje automático en los que se basan. El objetivo de este trabajo es hacer un estudio bibliográfico del campo de los modelos adversarios aplicados a la seguridad informática. Posteriormente, se elegirá un área aplicación, con sus respectivos conjuntos de datos y modelos representativos. Se realizarán experimentaciones en esta área y se analizarán los resultados.

% == Conocimientos Previos

% Los impartidos en el máster en Seguridad Informática además de básicos en aprendizaje automático.

% == Objetivos del TFM

% Estudio de los conceptos clave del campo del aprendizaje automático y del aprendizaje profundo en particular.
% Estudio de redes neuronales adversarias.
% Estudio de áreas de aplicación de estas redes en seguridad informática.
% Adaptación y ejecución de métodos de redes neuronales adversarias en un área de aplicación.
% Evaluar, comparar y analizar los resultados obtenidos

% == Metodología a Desarrollar

% Realizar una revisión bibliográfica del aprendizaje automático y del aprendizaje profundo
% Realizar una revisión bibliográfica del campo de los métodos de redes neuronales adversarias empleadas en seguridad informática.
% Estudiar las herramientas y bibliotecas existentes para el uso de redes neuronales adversarias.
% Estudiar y seleccionar un conjunto de datos para realizar la experimentación
% Diseñar la experimentación a realizar
% Adaptación y aplicación de métodos de redes adversarias al dataset seleccionado
% Analizar resultados y obtener conclusiones del trabajo realizado
% Escribir la memoria del trabajo realizado

En este capítulo explicaremos brevemente los objetivos que se quieren abordar en el proyecto, describiremos los objetivos principales como los específicos y su finalidad.



\section{Objetivos principales}

% El objetivo principal de este trabajo es hacer un estudio bibliográfico del campo de las redes neuronales adversarias aplicadas a la seguridad informática, búsqueda de vectores de ataque a modelos o a técnicas de seguridad clásicas, posibles formas de auditar y defender los distintos modelos. Además

% Nos centraremos en el estudio de los modelos generativos como \gls{GAN} o \gls{VAE}. Además de hacer un recorrido por la literatura actual de la \gls{IA} en la ciberseguridad.

El objetivo principal de este proyecto es el de realizar un estudio de las redes neuronales adversarias en el campo de la seguridad informática y su aplicación en la seguridad informática.

Realizar un estudio de la aplicación de redes neuronales en seguridad informática y analizar vectores de ataque a modelos de redes neuronales. Especialmente nos vamos a centrar en las redes neuronales adversarias y modelos generativos como son \gls{GAN} y \gls{VAE}, aunque existen otros modelos generativos, acotaremos la revisión bibliográfica a las \gls{GAN} únicamente y mencionaremos otros modelos con características similares.

Se adaptará y ejecutará métodos de redes neuronales adversarias en el área de seguridad, específicamente se ha planteado la temática de generar un modelo que sea capaz de evadir las medidas de seguridad clásicas de biometría. Es decir, usaremos una red neuronal adversaria para generar instancias falsas que sean capaces de evadir parcial o totalmente un sistema biométrico. El objetivo de esto es buscar si es posible evadir estas medidas de seguridad. Además, se deberá evaluar, comparar y analizar los resultados.

Se ha desarrollado un estudio de posibles vectores de ataque a algoritmos de biometría empleando estas arquitecturas y de como podemos optimizar la generación de estas instancias. Esto con el objetivo de generar instancias que puedan llegar a dar falsos positivos en sistemas de autenticación o identificación biométrica.

El objetivo de este proyecto es el de analizar si la seguridad informática puede ser comprometida desde distintos puntos, desde vulnerar un modelo que tenía fines legítimos o desde un atacante que utilice un modelo con fines maliciosos.

\section{Objetivos específicos}

% Como objetivos secundarios se ha planteado el análisis y creación de un marco de trabajo para la implementación de modelos de inteligencia artificial fiable siguiendo los estándares normativos, legislativos y éticos que propone la Unión Europea, con el objetivo de facilitar la implementación de modelos fiables en los distintos Estados miembros.


\begin{itemize}
    \item \textbf{Búsqueda de una visión general de la ciencia de datos actual}: Estudio del proceso KDD, aprendizaje profundo, redes neuronales, modelos y tareas. Nos enfocaremos en las redes neuronales adversarias aplicadas a la seguridad informática, centrándonos en redes generativas adversarias (\gls{GAN}) y brevemente en los (\gls{VAE}). Además, se analizará el estado normativo de estas aplicaciones dentro de la Unión Europea.
    
    % _TODO_: añadir este objetivo con este método.
    % OBJETIVO: Estudio de áreas de aplicación de estas redes en seguridad informática.
    % MÉTODO:   Realizar una revisión bibliográfica del campo de los métodos de redes neuronales adversarias empleadas en seguridad informática.
    
    \item \textbf{Análisis de vectores de ataque en redes neuronales.} Se analizarán los vectores de ataques específicos en la cadena de suministro que pueden ser comprometidos dentro de nuestro campo de estudio.
  
    \item \textbf{Implementar un modelo de red neuronal generativa adversaria que genere instancias sintéticas.} Con el fin de analizar la capacidad de las redes neuronales para evadir la seguridad, se creará un modelo con la tarea de generar instancias falsas que pueda provocar falsos positivos en sistemas de seguridad biométrica.
    
    \item \textbf{Optimizar la generación y eficiencia de instancias sintéticas.} Se buscará mejorar las técnicas de generación de instancias sintéticas, se explorará técnicas de optimización.
    
    \item \textbf{Evaluación del modelo mediante experimentos para extraer métricas.} Se deberá evaluar los resultados obtenidos por el modelo, analizando las métricas del modelo y analizando la evaluación de las instancias frente a algoritmos de extracción y coincidencia de características. 

    \item \textbf{Documentación y guía de instalación para la creación del modelo.} Se redactará la documentación de uso e instalación del modelo y las herramientas usadas durante el proyecto. Estos recursos estarán disponibles en el repositorio oficial del proyecto.
    % TODO Incluir cita del repositorio.
    
\end{itemize}