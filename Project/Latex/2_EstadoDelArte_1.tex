\section{Introducción}
\label{ch:2:section:introduction}

El proyecto que desarrollamos se encuentra en una frontera muy difusa de múltiples ramas del conocimiento, siendo muy interdisciplinar, se trata de una revisión de los ataques, seguridad y robustez a las redes neuronales, centrándonos en ataques adversariales.
Por lo que debemos explicar que es la ciencia de datos, el proceso \gls{KDD}, la inteligencia artificial generativa y la seguridad informática.

\begin{figure}[H]
    \centering
    \captionsetup{justification=centering}
    \centerline{\includesvg[width=0.75\columnwidth]{figures/ciencia-de-datos.drawio.svg}}
    \caption{Ciencia de datos como campo interdisciplinar.\newline{}Fuente: Elaboración propia.}
    \label{fig:ciencia-de-datos}
\end{figure}

Podemos dividir este trabajo en dos secciones muy relacionadas, la primera la inteligencia artificial y de segundo punto de importancia la seguridad de la información.
Desde sus inicios, la inteligencia artificial, aunque con buenos resultados en muchos campos de aplicación, resultaba en grandes fallos de seguridad, fiabilidad y robustez.
Por cómo están entrenadas las inteligencias artificiales (\acrshort{ANN}) tiene múltiples puntos de ataque que son susceptibles de ser atacados, los principales son los datos, las arquitecturas o los pesos.
Ya que alterando cualquiera de estos componentes de forma se verá enormemente afectada el comportamiento.

% Ciencia de la computación
% minería
% Machine Learning
% Deep Learning
% ANNs
% GANs

