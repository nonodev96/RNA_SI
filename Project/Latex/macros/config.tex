% ==== Introducir aquí el nombre del estudiante
\def\Estudiante{Antonio Mudarra Machuca}

% ==== Introducir aquí el nombre de los tutores. Si solo hay uno dejar las llaves de \TutorB vacías
\def\TutorA{Antonio Jesús Rivera Rivas}
\def\TutorB{María José del Jesus Díaz}
\def\Departamento{Departamento de informática}

% ==== Introducir aquí el título de completo y abreviado (para las cabeceras) del TFG
\def\TituloTFG{Redes neuronales adversarias en seguridad informática}
\def\TituloAbreviado{RNASI}

% ==== Introducir aquí el mes y año de presentación del TFG
\def\Fecha{septiembre de 2023}
\def\FechaPortada{septiembre, 2023}

% Configuración idioma
\renewcommand{\spanishtablename}{Tabla.}  % Título para las tablas
\renewcommand{\spanishcontentsname}{Tabla de contenidos}  % y los índices
\renewcommand{\spanishlistfigurename}{Lista de figuras}
\renewcommand{\spanishlisttablename}{Lista de tablas}

\renewcommand{\algorithmcfname}{Algoritmo}
\renewcommand{\listalgorithmcfname}{Lista de algoritmos}
\renewcommand{\lstlistingname}{Listado}
\renewcommand{\lstlistlistingname}{Lista de listados de código}

% Hoy en día la Inteligencia Artificial y sus aplicaciones están cada vez más implantadas en diferentes sistemas de nuestra sociedad. Dentro de esta disciplina el destacan el campo del Machine Learning (Aprendizaje Automático), donde como resultado de aplicar algoritmos de aprendizaje a datos se obtienen modelos que destacan por los resultados que obtienen. Gracias a su precisión estos modelos se encuentran desplegados en sistemas informáticos de gran importancia y que contralan procesos en diversos ámbitos de nuestra vida.

% Evidentemente estos sistemas pueden son vulnerables ataques de seguridad siendo los denominados modelos adversarios (basados normalmente en redes neuronales) los que suelen estar implicados tanto en estos ataques como en el posible robustecimiento de los sistemas atacados y por tanto de los modelos de aprendizaje automático en los que se basan. El objetivo de este trabajo es hacer un estudio bibliográfico del campo de los modelos adversarios aplicados a la seguridad informática. Posteriormente se elegirá un área aplicación, con sus respectivos conjuntos de datos y modelos representativos. Se realizarán experimentaciones en este área y se analizarán los resultados.

% Configuración tabularX
\newcolumntype{Y}{>{\centering\arraybackslash}X}



\hypersetup{
    colorlinks=true,
    linkcolor=blue,
    filecolor=magenta,
    urlcolor=cyan,
    pdftitle={\TituloTFG},
    pdfauthor={\Estudiante}
    bookmarks=true,
    bookmarksopen=true,
    pdfpagemode=FullScreen,
    breaklinks=true,
    citecolor=cyan,
}


\lstset{ %
    %language=delphi,                % the language of the code
    basicstyle=\linespread{0.7}\small\ttfamily,       % the size of the fonts that are used for the code
    numbers=left,                   % where to put the line-numbers
    numberstyle=\footnotesize\color{gray},  % the style that is used for the line-numbers
    stepnumber=1,                   % the step between two line-numbers. If it's 1, each line will be numbered
    numbersep=7pt,                  % how far the line-numbers are from the code
    backgroundcolor=\color{gray!5},  % choose the background color. You must add \usepackage{color}
    showspaces=false,               % show spaces adding particular underscores
    showstringspaces=true,         % underline spaces within strings
    showtabs=true,                 % show tabs within strings adding particular underscores
    frameround=fttt,
    frame=rtBL,                   % adds a frame around the code
    rulecolor=\color{black},        % if not set, the frame-color may be changed on line-breaks within not-black text (e.g. commens (green here))
    tabsize=4,                      % sets default tabsize to 2 spaces
    aboveskip=1em,
    captionpos=b,                   % sets the caption-position to bottom
    breaklines=true,                % sets automatic line breaking
    breakatwhitespace=false,        % sets if automatic breaks should only happen at whitespace
    title=\lstname,                 % show the filename of files included with \lstinputlisting;  also try caption instead of title
    keywordstyle=\bf\ttfamily,          % keyword style
    commentstyle=\color{black!60}\ttfamily,       % comment style
    stringstyle=\color{blue}\ttfamily,         % string literal style
    escapeinside={\%*}{*)}            % if you want to add a comment within your code
}

\lstset
{
    language=[LaTeX]TeX,
    breaklines=true,
    basicstyle=\tt\scriptsize,
    keywordstyle=\color{blue},
    identifierstyle=\color{gray},
    texcl=true
}
\lstset{
language=Python,
literate={á}{{\'a}}1
{ã}{{\~a}}1
{é}{{\'e}}1
{ó}{{\'o}}1
{í}{{\'i}}1
{ñ}{{\~n}}1
{¡}{{!`}}1
{¿}{{?`}}1
{ú}{{\'u}}1
{Í}{{\'I}}1
{Ó}{{\'O}}1
}

\renewcommand{\familydefault}{\sfdefault}
\setlength{\parskip}{1em}
\setlength{\headheight}{14.5pt}

% Configuración de encabezado y pie
\fancyhead[RE,LO]{{\color{gray}\Estudiante}}
\fancyhead[LE,RO]{{\color{gray}\TituloAbreviado}}
\fancyfoot[RE,LO]{{\color{gray}Escuela Politécnica Superior de Jaén}}
\fancyfoot[LE,RO]{{\color{gray}\thepage}}
\renewcommand{\footrulewidth}{1pt}


\definecolor{flashwhite}{rgb}{0.95, 0.95, 0.96}

% Emojis
% \setemojifont{EmojiOneMozilla}

% FancyHDR
\pagestyle{fancy}
\fancyhf{}

% algorithm2
\SetKwComment{Comment}{/* }{ */}
