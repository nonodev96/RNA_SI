\makeglossaries


% \addterm
% {sample}% label
% {slovene translation of the term}%
% {plural form of slovene translation of the term}%
% {english term}%
% {plural english term}%
% {slovene description of the term}
% Términos de seguridad clásica
\newglossaryentry{latex}        {name={latex},      description={Is a mark up language specially suited for scientific documents}}

\newglossaryentry{STRIDE}   {name={STRIDE},     description={Son las siglas de Suplantación de identidad, Manipulación, Repudio, Información, Denegación, Elevación de privilegios (Spoofing, Tampering, Repudiation, Information, Denial, Elevation)}}
\newglossaryentry{MITRE}    {name={MITRE},      description={La abreviatura de (Tácticas, Técnicas y Conocimiento Amplio de Enemigos) es un marco de trabajo para la evaluación de la seguridad en las organizaciones. \href{https://attack.mitre.org/}{Enlace}}}

% Generales
\newglossaryentry{AGI}          {name={AGI},        description={Inteligencia artificial general}}
\newglossaryentry{SDLC}         {name={SDLC},       description={Ciclo de vida del desarrollo de sistema}}
\newglossaryentry{SAST}         {name={SAST},       description={Las pruebas de seguridad de aplicaciones estáticas (SAST), también conocidas como Static Application Security Testing, de pruebas de seguridad}}
\newglossaryentry{DAST}         {name={DAST},       description={Las pruebas de seguridad de aplicaciones dinámicas (DAST), también conocidas como Dynamic Application Security Testing, de pruebas de seguridad}}
\newglossaryentry{IAST}         {name={IAST},       description={Las pruebas de seguridad de aplicaciones dinámicas (IAST), también conocidas como Interactive Application Security Testing, de pruebas de seguridad}}
\newglossaryentry{KDD}          {name={KDD},        description={El KDD viene de las siglas KDD (Knowledge Discovery in Databases), esto es el proceso utilizado para extraer de forma eficiente y automática \textbf{información útil} a partir de grandes volumenes de datos}}
%-> Terminos del libro


% Términos IA general
\newglossaryentry{ART-NN}   {name={ART (Adaptive Resonance Theory)},                description={Modelo de red neuronal que equilibra la adaptación a nueva información con la estabilidad frente a patrones familiares}}
\newglossaryentry{BP-NN}    {name={BP (Backpropagation)},                           description={Método utilizado en redes neuronales para calcular el gradiente y el cálculo de los pesos, es una abreviarura de ``propagación de errores hacia atrás''}}
\newglossaryentry{MLP}      {name={MLP (Multilayer Perceptron)},                    description={El Perceptron multicapa es una red neuronal artificial formada por capas local o totalmente conectadas}}

\newglossaryentry{ACC}      {name={ACC (Accuracy on a clean test/evaluation set)},  description={Precisión}}
\newglossaryentry{AI}       {name={AI (Artifical Intelligence)},                    description={La inteligencia artificial (IA), también conocida por su nombre inglés, Artificial Intelligence (AI), es una tecnología que trata de realizar las tareas y tomar las decisiones empresariales de forma automática y autónoma, aprendiendo de forma continua. \cite{glosario-tic-artificial-intelligence}}}
\newglossaryentry{ML}       {name={ML (Machine Learning)},                          description={El machine learning es la tecnología que permite que un sistema aprenda de forma continua. El sistema recibe un input, un humano reacciona y, así, la próxima vez que el sistema reciba ese input, sabrá cómo actuar sin necesidad de acudir al humano. \cite{glosario-tic-machine-learning}}}
\newglossaryentry{DL}       {name={DL (Deep Learning)},                             description={Deep learning (DL), también conocido como aprendizaje profundo, es un tipo de machine learning que se estructura inspirándose en el cerebro humano y sus redes neuronales. El aprendizaje profundo procesa datos para detectar objetos, reconocer conversaciones, traducir idiomas y tomar decisiones. Al ser un tipo de machine learning, esta tecnología sirve para que la inteligencia artificial aprenda de forma continua. \cite{glosario-tic-deep-learning}}}
\newglossaryentry{AL}       {name={AL (Active Learning)},                           description={Aprendizaje activo}}
\newglossaryentry{AUC}      {name={AUC (Area Under the (ROC) Curve)},               description={Área bajo la curva (ROC)}}
\newglossaryentry{PAUC}     {name={PAUC (Partial (ROC) Area Under the Curve)},      description={Área parcial bajo la curva (ROC)}}
\newglossaryentry{CS}       {name={CS (Cosine Similarity)},                         description={Similitud del coseno}}
\newglossaryentry{ROC}      {name={CS (Receiver Operating Characteristic)},         description={Característica Operativa del Receptor}}
\newglossaryentry{RL}       {name={RL (Aprendizaje reforzado)},                     description={Aprendizaje reforzado}}
%-> Terminos del libro


% Tipos de redes
\newglossaryentry{NN}       {name={NN (Neural Network)},                                                description={Red neuronal}}
\newglossaryentry{DNN}      {name={DNN (Deep Neural Network)},                                          description={Red neuronal profunda}}
\newglossaryentry{CNN}      {name={CNN (Convolutional Neural Network)},                                 description={Red neuronal convolucional}}
\newglossaryentry{GAN}      {name={GAN (Generative Adversarial Network)},                               description={Red generativa adversarial}}
\newglossaryentry{FNN}      {name={FNN (Feedforward Neural Network)},                                   description={Es una clase de redes neuronales}}
\newglossaryentry{RNN}      {name={RNN (Recurrent Neural Network)},                                     description={Es una clase de redes neuronales en la que las conexiones entre nodos forma un grafo dirigido a lo largo de una secuencia iterativa temporal. A diferencia de las FNN, las RNN puede usar sus pesos internos para procesar secuencias de entrada}}
\newglossaryentry{GNN}      {name={GNN (Graph Neural Network)},                                         description={Es una clase de redes neuronales especializada eb el procesamiento de datos que se puedan representar como gráficos}}
\newglossaryentry{LSTM}     {name={LSTM (Long Short-Term Memory)},                                      description={Memoria larga a corto plazo}}
\newglossaryentry{ResNet-n} {name={ResNet-n (Residual Neural Network architecture with $n$ layers)},    description={Arquitectura de red neuronal residual con $n$ capas}}
\newglossaryentry{LeNet-n}  {name={LeNet-n (Learnable Neural Network architecture with $n$ layers)},    description={Arquitectura de red neuronal aprendible con $n$ capas}}

\newglossaryentry{StyleGAN}  {name={StyleGAN (Learnable Neural Network architecture with $n$ layers)},    description={Arquitectura de red neuronal aprendible con $n$ capas}}
%-> Terminos del libro


% Estadistica
\newglossaryentry{OOD}  {name={OOD (Out-Of-Distribution)},                  description={Fuera de distribución}}
\newglossaryentry{OODD} {name={OODD (Out-Of-Distribution Detection)},       description={Detección fuera de distribución}}
\newglossaryentry{pdf}  {name={pdf (probability density function)},         description={función de densidad de probabilidad}}
\newglossaryentry{pmf}  {name={pmf (probability mass function)},            description={función de masa de probabilidad}}
\newglossaryentry{LR}   {name={LR (Logistic Regression)},                   description={Regresión logística}}
\newglossaryentry{BIC}  {name={BIC (Bayesian Information Criterion)},       description={Criterio de información bayesiano}}
\newglossaryentry{LEM}  {name={LEM (Local Error Maximizer)},                description={Maximizador de errores locales}}
\newglossaryentry{MAD}  {name={MAD (Median Absolute Deviation)},            description={Desviación absoluta mediana}}
\newglossaryentry{MAE}  {name={MAE (Mean Absolute Error)},                  description={Error absoluto medio}}
\newglossaryentry{MAP}  {name={MAP (Maximum a posteriori)},                 description={Máximo a posteriori}}
\newglossaryentry{MLE}  {name={MLE (Maximum Likelihood Estimation)},        description={Estimación de máxima verosimilitud}}
\newglossaryentry{MM}   {name={MM (Mixture Model)},                         description={Modelo de mezcla}}
\newglossaryentry{MSE}  {name={MSE (Mean-Squared Error)},                   description={Error medio cuadratico}}
\newglossaryentry{NB}   {name={NB (Native Bayes)},                          description={Bayes ingenuo}}
%-> Terminos del libro


% Ingenieria inversa
\newglossaryentry{RE}       {name={RE (Reverse-Engineering)},                           description={Ingeniería inversa}}
\newglossaryentry{RE-AP}    {name={RE-AP (Reverse-Engineering Additive Perturbation)},  description={Perturbación aditiva de ingeniería inversa}}
\newglossaryentry{RE-PR}    {name={RE-PR (Reverse-Engineering Patch Replacement)},      description={Reemplazo de parches de ingeniería inversa}}
\newglossaryentry{REA}      {name={REA (Reverse-Engineering Attack)},                   description={Ataques de ingeniería inversa}}
\newglossaryentry{RED}      {name={RED (Reverse-Engineering Defense)},                  description={Defensas frente ingeniería inversa}}
%-> Terminos del libro


% Análisis
\newglossaryentry{AD}       {name={AD (Anomaly Detection)}, description={Detección de anomalías}}
%-> Terminos del libro
\newglossaryentry{ADA}  {name={ADA (Anomaly Detection of TTE Attacks)}, description={Detección de anomalías en ataques de tipo TTE}}


% Ataques
\newglossaryentry{ASR}  {name={ASR (Attack Success Rate)},          description={Tasa de éxito del ataque}}
\newglossaryentry{BA}   {name={BA (Backdoor Attack (Trojan))},      description={Ataque de puerta trasera (troyano)}}
\newglossaryentry{DP}   {name={DP (Data Poisoning (attack))},       description={Envenanimiento de los datos}}
\newglossaryentry{BP}   {name={BP (Backdoor Pattern)},              description={Patrones de puerta trasera}}
\newglossaryentry{TTE}  {name={TTE (Test-Time Evasion (attack))},   description={Evasión en el tiempo de prueba}}
%-> Terminos del libro


% Defensas
%-> Terminos del libro
\newglossaryentry{PT}  {name={PT (Post-training)},   description={Posterior al entrenamiento}}


% Términos de internet
\newglossaryentry{LotL}  {name={LotL (living-off-the-land)},   description={Ataques a través de programas confiables.}}

% \newglossaryentry{AdvML}  {name={AdvML},      description={Aprendizaje automático adversarial}}
% \newglossaryentry{BIM}    {name={BIM},        description={Método iterativo básico}}
% \newglossaryentry{CNN}    {name={CNN},        description={Redes neuronales convolucionales}}
% \newglossaryentry{DDoS}   {name={DDoS},       description={Denegación de servicio distribuida}}
% \newglossaryentry{DT}     {name={DT},         description={Árbol de decisión}}
% \newglossaryentry{FFNN}   {name={FFNN},       description={Red neuronal directa}}
% \newglossaryentry{FGSM}   {name={FGSM},       description={Método de signo de gradiente rápido}}
% \newglossaryentry{FNR}    {name={FNR},        description={Tasa de falsos negativos}}
% \newglossaryentry{GAN}    {name={GAN},        description={Red Generativa Adversarial}}
% \newglossaryentry{GB}     {name={GB},         description={Refuerzo de gradiente}}
% \newglossaryentry{GMM}    {name={GMM},        description={Modelo de mezcla gaussiana}}
% \newglossaryentry{HIDS}   {name={HIDS},       description={IDS en casa}}
% \newglossaryentry{HSJ}    {name={HSJ},        description={Hop Skip Jump}}
\newglossaryentry{IDS}    {name={IDS},          description={Sistema de detección de intrusos}}
% \newglossaryentry{NIDS}   {name={NIDS},       description={IDS en red}}
% \newglossaryentry{IoT}    {name={IoT},        description={Internet de los objetos}}
% \newglossaryentry{JSMA}   {name={JSMA},       description={Ataque al mapa de saliencia basado en jacobianos}}
% \newglossaryentry{KNN}    {name={KNN},        description={Vecinos más próximos K}}
% \newglossaryentry{PGD}    {name={PGD},        description={Descenso gradual proyectado}}
% \newglossaryentry{SOA}    {name={SOA},        description={Estado de la técnica}}
% \newglossaryentry{VPC}    {name={VPC},        description={Clasificador de vectores de apoyo}}
% \newglossaryentry{SVM}    {name={SVM},        description={Máquina de vectores soporte}}
% \newglossaryentry{TAC}    {name={TAC},        description={Precisión total de la predicción}}
% \newglossaryentry{PAU}    {name={PAU},        description={Perturbaciones Adversariales Universales}}

\newacronym{nn}{NN}{Neural Network}
\newacronym{ann}{ANN}{Artificial Neural Network}
\newacronym{dnn}{DNN}{Deep Neural Network}
\newacronym{cnn}{CNN}{Convolutional Neural Network}
\newacronym{mlp}{MLP}{Multilayer Perceptron}
\newacronym{fnn}{FNN}{Feedforward Neural Network}
\newacronym{rnn}{RNN}{Recurrent Neural Network}
\newacronym{gan}{GAN}{Generative Adversarial Network}
\newacronym{gnn}{GNN}{Graph Neural Network}
\newacronym{dbn}{DBN}{Deep Belief Networks}
\newacronym{bp}{BP}{Backpropagation}
\newacronym{ai}{AI}{Artifical Intelligence}

\newacronym{lstm}{LSTM}{Long Short-Term Memory Network}
\newacronym{art-nn}{ART}{Adaptive Resonance Theory}



% a.s.:               almost surely (with probability one)
% ET:                 Expected Transferability

% CDF or cdf:         Cumulative Distribution Function
% GMM: Gaussian Mixture Model
% HC: High Confidence
% i.i.d.: independent and identically distributed
% JSD: Jensen Shannon Divergence
% KL: KulIback LeibIer divergence
% K NN: K Nearest Neighbors
% LC: Low Confidence
% PMM: Parsimonious Mixture Modeling
% PCA: Principal Component Analysis




%%% Términos
% -CS:                 Cosine Similarity
% -AUC:                Area Under the (ROC) Curve
% -PAUC: partial (ROC) Area Under the Curve
% -ACC:                Accuracy (on a clean test/evaluation set)
% -Al:                 Artificial Intelligence (often synonymous with a DNN)
% -AL:                 Active Learning
% -ML: Machine Learning
% -RL: Reinforcement Learning
% -ROC: Receiver Operating Characteristic

% PT: Post-Training
% FPR: False Positive Rate (fraction or percentage)
% TPR: True Positive Rate (fraction or percentage)

% SGD: Stochastic Gradient Descent
% SVM: Support Vector Machine
% TSC: Training Set Cleansing

% SIA: Source-class Inference Accuracy
% SVD: Singular Value Decomposition
% Al: eXplainabIe Al
% WB: White Box


%%% Tipos de redes
% -DNN:                Deep Neural Network
% -CNN:                Convolutional Neural Network
% -NN: Neural Network
% -GAN: Generative Adversarial Network
% -LSTM: Long Short-Term Memory (a recurrent NN)
% -ResNet-n: Residual Neural Network architecture with n layers
% -LeNet-n: Learnable Neural Network architecture with n layers


%%% Estadistica
% -BIC:                Bayesian Information Criterion
% -LR: Logistic Regression
% -MAD:    Median Absolute Deviation
% -MAE:    Mean Absolute Error
% -LEM: Local Error Maximizer
% -MAP:    Maximum a posteriori
% -MLE:    Maximum Likelihood Estimation
% -MM:     Mixture Model (or Maximum Margin in Chapter 9)
% -MSE:    Mean-Squared Error
% -NB:     Naive Bayes
% -OOD: Out-Of-Distribution
% -OODD: Out-Of-Distribution Detection
% -pdf: probability density function
% -pmf: probability mass function


%%% Ingenieria inversa
% -RE: Reverse-Engineering
% -RE-AP: Reverse-Engineering Additive Perturbation
% -RE-PR: Reverse-Engineering Patch Replacement
% -REA: Reverse-Engineering Attack
% -RED: Reverse-Engineering Defense


%%% Análisis
% -AD:                 Anomaly Detection (short name for 1-PT-RED in Chapter 6)


%%% Ataques
% -ASR:                Attack Success Rate
% -BP:                 Backdoor Pattern
% -DP:                 Data Poisoning (attack)
% -BA:                 Backdoor Attack (Trojan)
% TTE: Test-Time Evasion (attack), that is, adversarial input


%%% Defensas



