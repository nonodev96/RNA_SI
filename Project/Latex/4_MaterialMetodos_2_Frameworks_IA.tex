\section{Análisis de frameworks de Deep Learning}

Encontrar un \textit{framework} que cubra nuestras necesidades es una tarea compleja, ya que hay una gran variedad de estos que aun siendo respaldados sus desarrollos por grandes empresas son abandonados y puede no cumplir con nuestrar

En la siguiente sección sé muestran y se explican las características de los principales \textit{frameworks} de desarrollo de modelos de redes neuronales profundas más relevante del momento.

% =====================================
% ==== KERAS
% =====================================
\subsection{Keras}

\begin{figure}[H]
    \centering
    \includegraphics[width=7cm]{figures/assets/Logo_Keras.png}
    \caption{Keras}
    \label{fig:lib-keras}
\end{figure}

Keras \cite{chollet2015keras} es uno de los primeros \textit{frameworks} en la creación y ejecución de modelos de redes neuronales profundas, publicada en 2015, bajo la licencia \gls{LICENSE-MIT}, es de código abierto, soporta múltiples entornos de ejecución como \texttt{TensorFlow} \cite{Abadi_TensorFlow_Large-scale_machine_2015} o \texttt{CNTK} \cite{10.1145/2939672.2945397}.

Escrita en \texttt{python} el proyecto está estructurado en módulos independientes que suelen trabajar en conjunto, cuenta con soporte para capas neuronales, funciones de perdida, optimizadores, métricas, etc. Además, también permite extender estos módulos para adaptarlos a tus propias modificaciones. 

El objetivo principal de \texttt{Keras} es el de ser una interfaz intuitiva en vez de un framework completo, ya que utiliza se implementó el núcleo de \texttt{tensorflow}.

Características principales:
\begin{itemize}
    \item Interfaz sencilla para crear redes neuronales
    \item Definida por módulos que permiten agregar funcionalidades.
    \item Sencilla de extender sus características.
\end{itemize}

% \begin{table}[H]
%     \centering
%     \begin{tblr}{hlines,vlines,rows={valign=m},row{1}={c},colspec={XX}} 
%         \textbf{Ventajas}                           & \textbf{Desventajas}                             \\ 
%         Fácil de usar para principiantes            & Menos funciones avanzadas que otros frameworks   \\
%         Rápido de aprender para usuarios expertos   & Más restringido que frameworks avanzados         \\
%         Buena integración con frameworks            &                                                  \\
%     \end{tblr}
%     \caption{Keras: Ventajas y desventajas}
%     \label{tab:tensorflow-pros-cons}
% \end{table}


\begin{minipage}[t]{0.45\textwidth}
    \centering
    \textbf{Ventajas} \\[1ex]
    \begin{itemize}
        \item Fácil de usar para principiantes
        \item Rápido de aprender para usuarios expertos
        \item Es la base de otros frameworks
    \end{itemize}
\end{minipage}%
\hfill
\begin{minipage}[t]{0.45\textwidth}
    \centering
    \textbf{Desventajas} \\[1ex]
    \begin{itemize}
        \item Menos funciones avanzadas que otros frameworks
        \item Más restringido que frameworks avanzados
    \end{itemize}
\end{minipage}

% =====================================
% ==== TENSORFLOW
% =====================================
\subsection{TensorFlow}

\begin{figure}[H]
    \centering
    \includegraphics[width=7cm]{figures/assets/Logo_TensorFlow.png}
    \caption{TensorFlow}
    \label{fig:lib-tensorflow}
\end{figure}

TensorFlow \cite{Abadi_TensorFlow_Large-scale_machine_2015} es el \textit{framework} del grupo de investigación de inteligencia artificial de Google, es de código abierto y cuenta con la licencia \gls{LICENSE-APACHE-2}. Es uno de los más completos, cuenta con un amplio ecosistema, ya que se enfoca en proyectos empresariales. Proporciona una \gls{API} de \texttt{python} así como para los lenguajes de \texttt{C++}, \texttt{Haskell}, \texttt{Java}, \texttt{Go} y \texttt{Rust}, existen otras implementaciones de terceros de la \gls{API} de \texttt{TensorFlow}.

Además, se expande mediante implementaciones de alto rendimiento para dispositivos de bajos recursos, como es el caso de \texttt{tensorflow.js}, este permite ejecutar modelos ligeros directamente en un navegador web, ofrece distintos \gls{TENSORFLOW-BACKEND}. La \gls{API} es flexible y permite el uso de funciones de \texttt{Keras}.

Características principales:
\begin{itemize}
    \item Soporta un gran conjunto de herramientas y librerías para la construcción de redes neuronales, tanto convolucionales como recurrentes.
    \item Soporte para distintos tipos de hardware, tanto \gls{CPU}, \gls{GPU}, \gls{TPU}
\end{itemize}

% \begin{table}[H]
%     \centering
%     \begin{tblr}{hlines,vlines,rows={valign=m},row{1}={c},colspec={XX}} 
%         \textbf{Ventajas}                 & \textbf{Desventajas}                        \\ 
%         Multiplataforma                   & Curva de aprendizaje alta                   \\
%         Gran soporte a muchas tareas      & Grafo estático poco eficiente ante cambios  \\
%         Soporte y documentación           &                                             \\
%         Compatible con uso industrial     &                                             \\
%     \end{tblr}
%     \captionsetup{justification=centering}
%     \caption{TensorFlow: Ventajas y desventajas}
%     \label{tab:pros_cons}
% \end{table}

\begin{minipage}[t]{0.45\textwidth}
    \centering
    \textbf{Ventajas} \\[1ex]
    \begin{itemize}
        \item Multiplataforma
        \item Gran soporte a muchas tareas
        \item Soporte y documentación
        \item Compatible con uso industrial
    \end{itemize}
\end{minipage}%
\hfill
\begin{minipage}[t]{0.45\textwidth}
    \centering
    \textbf{Desventajas} \\[1ex]
    \begin{itemize}
        \item Curva de aprendizaje alta
        \item Grafo estático poco eficiente ante cambios
    \end{itemize}
\end{minipage}

% =====================================
% ==== PYTORCH
% =====================================
\subsection{PyTorch}

\begin{figure}[H]
    \centering
    \includegraphics[width=7cm]{figures/assets/Logo_PyTorch.png}
    \caption{PyTorch}
    \label{fig:lib-pytorch}
\end{figure}

PyTorch \cite{Ansel_PyTorch_2_Faster_2024} es uno de los \textit{frameworks} actuales más relevantes para el desarrollo de modelos de inteligencia artificial, es de código abierto y es desarrollada principalmente por \textit{Facebook's AI Research lab} (FAIR), cuenta con una licencia libre \gls{LICENSE-BSD}, fue lanzado en 2016. 

Cuenta con soporte para Windows, Linux y MacOS, cuenta con soporte para gestores de paquetes de \texttt{conda} y \texttt{pip}, además PyTorch distribuye su librería para otros sistemas (C++) mediante \texttt{LibTorch}, da soporte para tarjetas gráficas NVIDIA con tecnología \texttt{CUDA[11.8, 12.1, 12.4]}, tarjetas gráficas \texttt{AMD ROCm[6.2]} o \gls{CPU}.

Sus principales ventajas es que es la fusión de varios proyectos con otras organizaciones, por lo que existe una compatibilidad entre estos \textit{Frameworks}. Es compatible con otras librerías de ciencias de datos, \textit{numpy}, \textit{scipy}, \textit{panda}.

% Características 
Soporta la ejecución y entrenamiento en \gls{GPU} y \gls{TPU} y la distribución entre múltiples tarjetas gráficas, cuenta con una \gls{API} disponible para otros lenguajes, además de modelos pre-entrenados, conjuntos de datos o ejemplos de arquitecturas comunes en la literatura actual.

% Ecosistema
PyTorch cuenta con un gran ecosistema, estos son otros proyectos que expanden la librería y permite realizar, desde visualización de los procesos, entrenamientos o resultados, exportar los modelos a otros entornos de ejecución, etc.

\textbf{Características principales}:
\begin{itemize}
    \item PyTorch trabaja con un grafo dinámico haciendo que el desarrollo y experimentación sea más rápido.
    \item Programación imperativa permite construir modelos más sencillos.
    \item Soporte para \gls{GPU}, \gls{TPU}, etc.
\end{itemize}

% \begin{table}[H]
%     \centering
%     \begin{tblr}{hlines,vlines,rows={valign=m},row{1}={c},colspec={Q[m]X}} 
%         \textbf{Ventajas}                          & \textbf{Desventajas}                       \\ 
%         Fácil para principiantes                   & Menos funciones que TensorFlow             \\
%         Adaptable para desarrollo e investigación  & Menos opciones para despliegue             \\
%         Compatible con grafo dinámico              &                                           \\
%     \end{tblr}
%     \caption{PyTorch: Ventajas y desventajas}
%     \label{tab:pytorch-pros_cons}
% \end{table}

\begin{multicols}{2} % Inicia el entorno de dos columnas
    \centering
    \textbf{Ventajas} \\[1ex]
    \begin{itemize}
        \item Fácil para principiantes
        \item Adaptable para desarrollo e investigación
        \item Compatible con grafo dinámico
    \end{itemize}
    
    \columnbreak % Cambia a la siguiente columna

    \centering
    \textbf{Desventajas} \\[1ex]
    \begin{itemize}
        \item Menos funciones que TensorFlow
        \item Menos opciones para despliegue
    \end{itemize}
\end{multicols}

\begin{table}[H]
    \centering
    \begin{tblr}{hlines,vlines,rows={valign=m},colspec={Q[m]X}} 
        \textbf{Módulo}             & \textbf{Descripción}                                                                      \\ 
        \texttt{torch}              & Biblioteca principal para el cálculo numérico y la manipulación de tensores.              \\ 
        \texttt{torch.nn}           & Proporciona herramientas para construir redes neuronales y capas.                         \\ 
        \texttt{torch.autograd}     & Soporte para la diferenciación automática para la optimización del entrenamiento.         \\ 
        \texttt{torch.optim}        & Optimizadores, como SGD y Adam, para ajustar los parámetros de los modelos.               \\ 
        \texttt{torch.utils.data}   & Herramientas para el manejo de datos y su carga durante el entrenamiento.                 \\
    \end{tblr}
    \caption{Núcleo de PyTorch}
    \label{tab:pytorch-modules}
\end{table}

\begin{table}[H]
    \centering
    \begin{tblr}{hlines,vlines,rows={valign=m},colspec={Q[m]X}} 
        \textbf{Tecnologías}    & \textbf{Descripción}                                                                                      \\ 
        \texttt{torch.cuda}     & Permite el uso de GPUs NVIDIA para acelerar los cálculos de tensores                                      \\
        \texttt{torch.onnx}     & Herramientas para exportar modelos de PyTorch al formato ONNX para interoperabilidad con otros frameworks \\
    \end{tblr}
    \caption{Tecnologías de PyTorch}
    \label{tab:pytorch-gpu}
\end{table}

\begin{table}[H]
    \centering
    \begin{tblr}{hlines,vlines,rows={valign=m},colspec={Q[m]X}} 
        \textbf{Librería}       & \textbf{Descripción}                                                                                  \\ 
        \texttt{torchvision}    & Conjunto de bibliotecas para tareas de visión por computadora, con modelos preentrenados y utilidades \\ 
        \texttt{torchaudio}     & Biblioteca para trabajar con datos de audio, con soporte para diversas operaciones de audio           \\ 
        \texttt{torchtext}      & Biblioteca para la carga, preprocesamiento y uso de datos de texto en tareas de NLP                   \\
    \end{tblr}
    \caption{Librería de PyTorch}
    \label{tab:pytorch-lib}
\end{table}



% =====================================
% ==== PyTorch + PyTorch Lightning
% =====================================
\subsection{PyTorch Lightning}

\begin{figure}[H]
    \centering
    \includegraphics[width=7cm]{figures/assets/Logo_Lightning.png}
    \caption{PyTorch Lightning}
    \label{fig:pytorch-lightning}
\end{figure}

PyTorch Lightning \cite{Falcon_PyTorch_Lightning_2019} es un framework que expande y simplifica PyTorch, este permite simplificar el código de PyTorch a una arquitectura de objetos con el objetivo de realizar módulos más reutilizables.

Este framework permite simplificar y abstraer el código, controla la gestión de memoria y de dispositivos, permite crear y pruebas o depurar de forma más sencilla. 

Cuenta con una extensa \texttt{API}, cuenta con gestión de modelos en dispositivos \textit{Accelerators}, ejecución de funciones por eventos mediante \texttt{Callbacks}, un \gls{CLI} para la gestión por terminal, gestión de \texttt{logs}, control del entrenamiento mediante la clase \texttt{Trainer}, gestión automática de hiperparámetros mediante \texttt{Tuner} y muchas más utilidades.


\subsection{Otras alternativas, frameworks de deep learning}

Se han analizado muchos frameworks que cumplen parcialmente nuestras necesidades, algunas de las alternativas son las siguientes. Muchos de estos proyectos siguen en desarrollo y aumentando sus capacidades, aunque otros o han llegado a su objetivo o han sido abandonados parcial o totalmente.

\begin{table}[H]
    \centering
    \begin{tblr}{hlines,vlines,rows={valign=m},colspec={XX}} 
        \textbf{Framework}      & \textbf{Especialización}  & \textbf{Empresa}              & \textbf{Estado}   \\ 
        MXNet                   & Generalista               & F. APACHE y Amazon            & En desarrollo     \\ 
        CAFFE                   & Visión por computador     & Berkeley AI Research          & Abandonado        \\ 
        CNTK                    & Generalista               & Microsoft                     & Abandonado        \\ 
        LangChain               & LLMs                      & LangChain, Inc                & En desarrollo     \\ 
        OpenNN                  & Generalista               & Artelnics                     & Abandonado        \\ 
        Deeplearning4j          & Generalista               & Kondiut and contributors      & En desarrollo     \\ 
        Chainer                 & Generalista               & Seiya and contributors        & Abandonado        \\ 
    \end{tblr}
    \caption{Frameworks de deep learning}
    \label{tab:other-frameworks}
\end{table}

