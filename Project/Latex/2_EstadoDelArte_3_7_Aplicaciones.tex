\section{Aplicaciones de las redes neuronales en la seguridad informática}
% Estudio de áreas de aplicación de estas redes en seguridad informática.
% Realizar una revisión bibliográfica del campo de los métodos de redes neuronales adversarias empleadas en seguridad informática.

% region subsection Las redes neuronales para la seguridad informática

Como hemos tratado previamente, existen distintas aplicaciones que se le puede dar a las redes neuronales para segurizar el funcionamiento de un sistema o proteger a un usuario frente ataques maliciosos.

En la actualidad existen muchos tipos de ataque informáticos, entre estos destacan malware, phishing, spam, man in the middle, denegación de servicio, inyección SQL, DNS \textit{Tunneling}, \textit{botnets}, o vulnerabilidades \textit{zero-day} \cite{cisco-attacks}, estos son solo una clasificación rápida de todos los tipos de ataques que se pueden sufrir.

La \gls{IA} ofrece numerosas ventajas en la recopilación de información, toma de decisiones y autonomía de sistemas, podemos usar modelos de inteligencia artificial para mejorar la seguridad clásica de los sistemas de información. Esto permite mejorar la eficiencia de los métodos de seguridad empresarial, seguridad ciudadana, etc.

Estos campos han impulsado una carrera armamentística entre defensores y atacantes, además del desarrollo de códigos éticos y regulaciones internacionales en el ámbito militar.

Algunas herramientas que han llegado al ámbito empresarial para mejorar sus sistemas de seguridad, destacan:

\href{https://cylance.com/}{Cylance} Es una de las principales empresas que proporcionan herramientas de seguridad informática que emplean inteligencia artificial.
\begin{itemize}
    \item CylancePROTECT: plataforma de protección de puntos finales, especializada en identificar y neutralizar malware. Mediante el análisis de archivos y la ejecución de modelos predictivos. Es versátil ya que emplea los modelos predictivos para anticipar el comportamiento de malware desconocido.
    \item CylanceOPTICS: herramienta de detección y respuesta, permite la visualización de las etapas de un ataque, facilita la respuesta rápida y automatizadas basadas en reglas a las amenazas identificando, minimiza el impacto sin bloquear el resto de sistemas no comprometidos.
\end{itemize}

\href{https://www.fortinet.com/lat/products/fortiaiops}{Fortinet} Es una de las empresas con más recorrido en la seguridad informática, especializada en operaciones TI. Cuenta con \texttt{FortiAI}, que es un dispositivo hardware y software para empresas, este cuenta con millones de funciones de malware entrenadas, detecta y analiza amenazas en tiempo real, además de manejar un gran volumen de tráfico de red en tiempo real. Estas soluciones todo en uno vienen acompañadas con decenas de soluciones adicionales. Asume tareas de \gls{IDS}, \gls{SIEM} y \gls{SOAR}.



% endregion subsection Las redes neuronales para la seguridad informática

% TODO: aplicaciones reales de ciberseguridad que usan inteligencia artificial 
{
\Huge
\color{red} 
TODO: 
}
% añadir lista de posibles tareas que pueden realizar las redes neuronales por la seguridad y ejemplos de aplicaciones en producción.